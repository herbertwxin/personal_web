%! Author = herbertxin
%! Date = 1/20/25

\documentclass[twocolumn, fleqn]{article}
\usepackage[margin=1in]{geometry}
\usepackage{fancyhdr}
\usepackage{graphicx}
\usepackage{amsmath}
\usepackage{hyperref}
\usepackage{enumitem}
\usepackage{amssymb}
\usepackage{tcolorbox}
\usepackage{float}



% Adjust column separation and add vertical rule
\setlength{\columnsep}{25pt} % Increase space between columns
\setlength{\columnseprule}{0.6pt}
\setlength{\jot}{6pt} % Set the space between equation lines to 6pt
\setlength{\parskip}{3pt}

% Header and Footer Settings
\pagestyle{fancy}
\fancyhf{}
\lhead{Core Micro II}
\chead{Producer Theory}
\rhead{Page \thepage}
\renewcommand{\headrulewidth}{0.4pt}
\newtcolorbox{note}[1][]{
	enhanced,
	colback=white,            % Background color of the box content
	colframe=black,           % Border color
	coltitle=white,           % Title text color
	colbacktitle=black,       % Background color of the title area
	fonttitle=\bfseries,      % Bold font for the title
	boxrule=1pt,              % Border thickness
	title=NOTE,               % Title of the box
	sharp corners,            % Square corners
	before skip=10pt,         % Space before the box
	after skip=10pt,          % Space after the box
	left=5pt,                 % Left padding
	right=5pt,                % Right padding
	top=5pt,                  % Top padding
	bottom=5pt,               % Bottom padding
	width=\columnwidth,       % Box width matches column width
	#1                        % Allows for optional parameters
}
\newcommand{\argmax}{\operatorname*{arg\,max}}
\newcommand{\argmin}{\operatorname*{arg\,min}}

\begin{document}

	\title{Producer Theory}
	\author{Herbert W. Xin}
	\date{\today}
	\maketitle

	\tableofcontents
	\thispagestyle{fancy}
	
	\section{Technology}
		A \textbf{production technology} can be described as a function that maps multiple inputs into single outputs.
		\[ y = f(\bold x)\]
		where $\bold x$ is a vector of inputs and $y$ is the single output.
		
		
		A \textbf{production plan} $(y, x)$, on the other hand, is a technologically efficient allocation where there does not exist a production plan $(y', x') \neq (y, x)$ such that $y' \geq y, x_i' \leq x_i, \forall i$.
		
		An \textbf{isoquant} is the set of all positive input bundles that allow the firm to produce exactly $y$.
		\[Q(y) = \lbrace x \in \mathbb R_+^h | f(x)=y \rbrace \]
		
		Now, the return to scale of a production function can be easily defined using degrees of homogeneity 
		\begin{itemize}
			\item IRS when homogeneous of degree greater than one
			\item CRS when homogeneous of degree one
			\item DRS when homogeneous of degree less than one
		\end{itemize}
		
		The \textbf{marginal product} and \textbf{average product} of input $x_i$ are 
		\[ MP = \frac{\partial f(x)}{\partial x_i}, AP = \frac{f(x)}{x_i}\]
		
		Now, an important concept here is \textbf{marginal rate of transformation} (MRT), or marginal rate of technical substitution (MRTS). I follow the notation of MWG here. 
		MRT measures the tradeoff between two goods while keeping output constant, i.e. 
		\[MRT_{ij}(\bar y) = \frac{\partial f(\bar y) / \partial x_i}{\partial f(\bar y) / \partial x_j}\]
		
		To see this, we observe 
		\[\frac{\partial f(\bar y)}{\partial x_i} d y_i + \frac{\partial f(\bar y)}{\partial x_j}dy_j=0\] 
		is the change in $x_i$ and $x_j$ such that $\bar y$ stays constant, which is precisely 
		\[\left|\frac{d x_j}{d x_i}\right| = \frac{\partial f(\bar y) / \partial x_i}{\partial f(\bar y) / \partial x_j} = MRT_{ij}\]
		
		The \textbf{elasticity of substitution} is 
		\[\sigma = \frac{\partial \ln(x_j/x_i)}{\partial \ln MRT(x_i, x_j)}= \frac{\partial \ln(x_j/x_i)}{\partial \ln (\frac{\partial f}{\partial x_i}/\frac{\partial f}{\partial x_j})}\]
		which represents the change in input ratio given a change in MRT. High $\sigma$ thus represents the two inputs are highly substitutable given a small change in MRT.
		
		Graphically, MRT represents the slope of the isoquant while elasticity represents the curvature. When $\sigma=0$, we have perfect complement, i.e. inputs cannot be substitute for each other, and we have Leontieff isoquant. When $\sigma = \infty$ we have perfect substitute and the isoquant is a straight. 
		
	\section{Profit Maximization}
		The \textbf{unconditional factor demand} is calculated as 
		\[x(p,w) = \argmax _x \left[ p \cdot f(x)-wx \right]\]
		where $\argmax$ outputs the arguments that maximizes the function.
		
		The \textbf{supply function} is then 
		\[y(p,w) = f(x(p,w))\]
		
		Now the \textbf{law of suppply} says: $\frac{\partial y}{\partial p} \geq 0$, more formally,
		\[(p''-p')[y(p'', w)-y(p',w)]\geq 0\]
		
		Similarly, the \textbf{law of demand} gives $\frac{\partial x}{\partial p} \leq 0$, which is 
		\[(w''-w')[x(p, w'')-x(p,w'')]\leq 0\]
		
		The supply function $y(p,w)$ and factor demand $x(p,w)$ are homogeneous of degree 0, as 
		\[pf(x) - wx \geq pf(x')-wx'\]
		scaling the inequality by $\lambda$, i.e. both price and factor price got scaled by $\lambda$, the inequality still holds
		\[\lambda pf(x) - \lambda wx \geq \lambda pf(x')- \lambda wx'\]
		meaning the terms related to $x$ has not changed, otherwise it would become $\lambda x$, and this implies HD0. Since the factor demand is HD0, then the supply function is also HD0, as $y(x(p,w))$.
		
		The profit function $\pi (p,w)$, on the other hand, is homogeneous of degree 1, as
		\[\pi(\lambda p,\lambda w)=\lambda pf(x) - \lambda wx = \lambda \pi(p,w)\]
		
		The profit function is also convex in $(p,w)$.
		
		\subsection{Envelope theorem}
		
			In short, the \textbf{envelope theorem} says changing the parameter only causes direct effect, not indirect effect through the argument at optimal.
			
			Consider the maximization problem 
			\[V(\alpha) = \max_x u(x,\alpha)\]
			Let $x^\ast (\alpha) = \argmax_x u(x,\alpha)$, since we know $x^\ast(\alpha)$ is the optimal, it must be true that 
			\[\frac{\partial u(x^\ast(\alpha), \alpha)}{\partial x}=\frac{\partial u^\ast}{\partial x}=0\]
			To see the envelope theorem, take the derivate of $V(\alpha)=u(x^\ast (\alpha), \alpha)$, i.e. at the optimal.
			\[\frac{\partial V}{\partial \alpha}=\frac{\partial u^\ast}{\partial x} \frac{\partial x^\ast}{\partial \alpha}+\frac{\partial u}{\partial \alpha}=\frac{\partial u}{\partial \alpha}\] 
			
			
			Now, the profit function at optimal is 
			\[\pi^\ast (p,w) = pf(x^\ast(p,w)) - wx^\ast(p,w) \]
			and by optimality condition
			\[\frac{\partial x^\ast(p,w)}{\partial p}=0\]
			Thus, applying the envelope theorem to the profit function gives the hotelling's lemma:
			\[\frac{\partial \pi^\ast (p,w)}{p}= f(x^\ast(p,w)) = y(p,w) \geq 0\]
			Similarly, 
			\[\frac{\partial \pi^\ast (p,w)}{w_i}=-x^\ast_i(p,w) \leq 0\]
			
			
	\section{Cost Minimization}
	
		With cost minimization problem we solve
		\[ \min_{x} w x, \text{ subject to } f(x) \geq y \]
		
		The resulting first-order necessary conditions are
		\begin{align*}
			y &= f(x^\ast)\\
			w_i &\geq \lambda \frac{\partial f(x^\ast)}{\partial x_i}
		\end{align*}
		with equality if $x_i^\ast >0$, these first order conditions are sufficient if $f(x)$ is quasi-concave.
		
		In the case of no corner solutions, the necessary conditions can be written as 
		\begin{align*}
			MRTS = \left| \frac{dx_i}{dx_j}\right| = \frac{\partial f(x^\ast)/\partial x_j}{\partial f(x^\ast)/\partial x_i} = \frac{w_j}{w_i}
		\end{align*}
		and 
		\[y = f(x^\ast)\]
		which is the exactly the tangency of isoquant and isocost line.
		
		We define the resulting factor demand as conditional factor demand
		\[z(w,y) = \argmin_x (w)x \ \text{ subject to } f(x) = y\]
		
		The conditional factor demand is \textbf{HD0}
		
		The corresponding cost function is 
		\[c(w,y) = \sum_{i=1}^{h} w_i z_i (w,y) = w z(w,y)\]
		
		\subsection{Shepard's Lemma}
		If $z(w,y)$ is single valued and $c(w,y)$ differentiable w.r.t $w$, then 
		\[\frac{\partial c(w,y)}{\partial w_i} = z_i (w,y)\]
		
		To proof it, notice that the cost function is 
		\[c(w,y) = \sum_{i=1}^{h} w_i z_i (w,y) \]
		differentiate w.r.t. $w_i$ and apply the envelope theorem yields the result.
		
		Similarily, the Lagrangian multiplier of the cost minimization problem is the marginal cost of output
		\[\frac{\partial c(w,y)}{\partial y} = \lambda(w,y)\]
		Again, this results from taking derivate w.r.t $y$ and apply envelope theorem on the cost function.
		\[c(w,y) = \min_x wx - \lambda(w,y)(f(x)-y)\]
		
		The cost function is also 
		\begin{enumerate}
			\item non-decreasing in y
			\item HD1 in $w$
			\item concave in $W$
		\end{enumerate}
		
		\subsection{Properties of technology}
		Some other properties includes
		\begin{itemize}
			\item If $f(x)$ is homogenous of degree 1 (CRS), then conditional factor demands $z(w,y)$ are also homogenous of degree 1 in $y$.
			\item If $f(x)$ is homogenous of degree 1 (CRS), then conditional factor demands $c(w,y)$ are also homogenous of degree 1 in $y$.
			\item If $f(x)$ exhibits IRS, then the cost function $c(w,y)$ is concave in $y$.
			\item $f(x)$ with IRS has decreasing marginal cost function and average cost function such that \[\frac{\partial c(w,y)}{\partial y}\leq \frac{c(w,y)}{y}\]
			\item If $f(x)$ exhibits DRS, then cost function $c(w,y)$ is convex in $y$
			\item $f(x)$ with IRS has increasing marginal cost function and average cost function such that \[\frac{\partial c(w,y)}{\partial y}\geq \frac{c(w,y)}{y}\]
		\end{itemize}
		
		\subsection{Profit maximization with Cost function}
		The maximization problem is now 
		\begin{align*}
			\max_y py-c(w,y)
		\end{align*}
		
		Necessary FOCs are
		\[p \leq \frac{\partial c(w,y^\ast)}{\partial y}\]
		with equality if output is positive
		
		Sufficient SOC for local maximum is 
		\[\frac{\partial^2 c(w,y^\ast)}{\partial y^2}>0\]
		
		For all inputs $i,j$ such that $x_i>0, x_j>0$,
		\[\frac{w_i}{MP_i} = \frac{w_j}{MP_j} = MC\]
		holds at equilibrium.
		
		\section{Equilibrium}
		\subsection{Aggregation}
		The total production is the sum of production from each firm
		\begin{align*}
			y(p,w) = \sum_{j=1}^{J} y^j(p,w)
		\end{align*}
		
		The aggregate producer then solves
		\begin{align*}
			&\max_{x^1, \ldots,x^J} \sum_{j=1}^J p f^j(x_j) - \sum_{j=1}^J w x_j \\
			=&\sum_{j=1}^{J} \max_{x_j} (pf^j(x_j)-wx_j)
		\end{align*}
		
		where $f(x)$ is the production technology.\\
		
		An allocation is feasible if 
		\begin{align*}
			\sum_i x_l^i \leq w_l + \sum_j y_l^j - \sum_j z_l^j
		\end{align*}
		where $x_l^i$ is the consumption of good $l$ by $i$. $w_l$ is the endowment of good $l$. $y_l^j$ is the production of $l$ by firm $j$, and $z_l^j$ is the input $l$ used by firm $j$.
		
		\subsection{Equilibrium}
		The condition for competitive equilibrium is 
		\begin{itemize}
			\item Firms maximize profits
			\item Consumer maximize utility
			\item Market clears
		\end{itemize}
		Note both firms and consumers take price $p^\ast$ as given\\
		
		\subsubsection{Firm's problem}
		Firm solves the profit maximization problem
		\begin{align*}
			\max_{y^j, z^j} p^\ast y^j - p^\ast z^j
		\end{align*}
		subject to technological feasibility
		
		where $\pi^{j \ast}$ denotes the maximum profit of $j$
		
		\subsubsection{Consumer's problem}
		The consumer solves the problem of 
		\begin{align*}
			\max_{x^i} u^i(x^i)
		\end{align*}
		subject to the budget constraint
		\[p^\ast x^i \leq p^\ast w^i +\sum_j \theta_j^i \pi^{j\ast}\]
		where $\theta_j^i$ is $i$'s share of firm $j$
		
		\subsubsection{Market clearing}
		for each good $l = 1, \ldots, L$
		\begin{align*}
			\sum_i x_l^{i\ast} + \sum_j z_l^{j \ast} = w_l + \sum_j y_l^{j \ast}
		\end{align*}
		
		\subsection{First Welfare Theorem}
		Assume preferences are locally non-satiated. If \((x^*, y^*, z^*, p^*)\) is a competitive equilibrium, then the associated allocation \((x^*, y^*, z^*)\) is \textcolor{orange}{\textbf{Pareto optimal}}.
		
		To state this mathematically, Pareto efficient means there does not exist any feasible allocation $(x', y', z')$ such that 
		\[ u^{\hat{i}}(x^{\hat{i}'}) > u^{\hat{i}}(x^{\hat{i}\ast}) \tag{4.3.1}\]
		for at least one $hat{i}$ and
		\[ u^i(x^{i'}) \geq u^i(x^{i\ast}) \tag{4.3.2}\]
		for all $i$, which can be proved by contradiction.
		
		For (4.3.1), if such feasible allocation exits, then $x^{\hat{i}\ast}$ is violates the utility maximization condition.
		
		Thus, it must be the case 
		\[p^\ast x^{\hat{i}'}> p^\ast w^{\hat{i}} +\sum_j \theta_j^{\hat{i}} \pi^{j\ast} \tag{4.3.3}\]
		
		For (4.3.2), it must the case that 
		\[p^\ast x^{i'} \geq p^\ast w^{i} +\sum_j \theta_j^{i} \pi^{j\ast}\tag{4.3.4}\]
		Otherwise, by local non-satiated there is another bundle that is also feasible and preferred to $x^{i'}$.
		
		Sum (4.3.3) and (4.3.4), i.e. over all $i$, we get 
		\[
		p^* \sum_i x^{i'} > p^* \omega + \sum_j \pi^{j*}\]
		
		Also, we know that 
		\[
		\pi^{j*} = p^* y^{j*} - p^* z^{j*} \geq p^* y^{j'} - p^* z^{j'}
		\]
		This implies 
		\[
		p^* \sum_i x^{i'} > p^* \omega + \sum_j \pi^{j*} \geq p^* \omega + \sum_j \big(p^* y^{j'} - p^* z^{j'}\big)
		\]
		
		Now we can rewrite it in terms of each good 
		\[
		\sum_{\ell} p^*_{\ell} \sum_i x^{i'}_{\ell} > \sum_{\ell} p^*_{\ell} \omega_{\ell} + \sum_{\ell} \sum_j \big(p^*_{\ell} y^{j'}_{\ell} - p^*_{\ell} z^{j'}_{\ell}\big)
		\]
		
		Sum across all goods 
		\[
		p^* \sum_i x^{i'} > p^* \omega + \sum_j \big(p^* y^{j'} - p^* z^{j'}\big) \tag{4.3.5}
		\]
		
		
		Now, by the feasibility condition, we have 
		\[
		\sum_i x^{i'}_{\ell} \leq \omega_{\ell} + \sum_j y^{j'}_{\ell} - \sum_j z^{j'}_{\ell}
		\]
		
		Multiply by $p^\ast_l$ and sum across goods gives
		\[
		p^* \sum_i x^{i'} \leq p^* \omega + \sum_j \big(p^* y^{j'} - p^* z^{j'}\big) \tag{4.3.6}
		\]
		
		Now clearly (4.3.5) and (4.3.6) violates each other and completes the proof.
		
		\section{Partial Equilibrium Analysis}
		\subsection{Consumer's optimization}
		In this section, we uses a quasi-linear utility function:
		\begin{align*}
			u^i(m^i,x^i)&=m^i + \phi^i (x^i)\\
			\phi' >0,  \quad \phi''<0, &\quad \phi' \rightarrow 0, \text{ as } x^i \rightarrow \infty
		\end{align*}		
		where $x$ is the consumption good and $m$ is the numeraire, money, in this case.
		
		The quasi-linear preference meant there is no wealth effect and agent can consume negative amount of numeraire. 
		
		We take $p$ to be the price of good $x$ and normalized price of $m$ to 1. Consumers haver endowment $\omega^i$ in numeraire and own shares $\theta_j^i$ of firm $j$.
		
		So the consumer solves 
		\begin{align*}
			\max_{x^i, m^i} m^i + \phi^i(x^i)
		\end{align*}
		subject to 
		\begin{align*}
			m^i + p^\ast x^i \leq \omega^i + \sum_j \theta^i_j \pi^{j\ast}
		\end{align*}
		
		At optimal, the budget constrain holds with equality, i.e. 
		\begin{align*}
			m^i = \omega^i + \sum_j \theta^i_j \pi^{j\ast} - p^\ast x^i
		\end{align*}
		Substitute this into the maximization problem 
		\begin{align*}
			\max_{x^i} \phi^i (x^i) - p^\ast x^i + \omega^i + \sum_j \theta^i_j \pi^{j\ast}
		\end{align*}
		The necessary and sufficient FOC is then 
		\begin{align*}
			\phi^{i \prime}(x^{i\ast})\leq p^\ast
		\end{align*}
		
		\subsection{Firm's optimization}
		The firm's profit maximization is given by
		\begin{align*}
			\pi^{j\ast} = \max_{q^j} p^\ast q^j - c^j (q^j)
		\end{align*}
		where $p^\ast$ is given and $c^j(q^j)$ is the firm's cost function. The necessary and sufficient FOC is 
		\begin{align*}
			p^\ast \leq c^{j\prime} (q^{j\ast})
		\end{align*}
		
		\subsection{Market clearing}
		The consumption market clears when 
		\begin{align*}
			\sum_i x^{i \ast} = \sum_j q^{j^\ast}
		\end{align*}
		
		Now consumption market clearing, combined with consumer and firm's maximization problem implies to the numeraire clearing. From the consumer's budget constraint
		\begin{align*}
			m^i + p^\ast x^i \leq \omega^i + \sum_j \theta^i_j \pi^{j\ast}
		\end{align*}
		summing over individuals and plugging in the definition of profits 
		\begin{align*}
			\sum_i m^i + p^\ast \sum_i x^i = \omega + \sum_j (p^\ast q^{j \ast}-c^j (q^{j\ast}))
		\end{align*}
		With the consumption market clearing condition, we get
		\begin{align*}
			\sum_i m^{i\ast} + \sum_j c^j (q^{j \ast})=\omega
		\end{align*}
		which is the market clearing for numeraire.
		
		The aggregate market clearing condition is then 
		\begin{align}
			p^\ast &\leq c^{j\prime} (q^{j\ast})\\
			\phi^{i \prime}(x^{i\ast})&\leq p^\ast\\
			\sum_i x^{i \ast} &= \sum_j q^{j^\ast}
		\end{align}
		This aligns with the competitive equilibrium condition discussed earlier. Also, note that these equilibrium condition don't depend on endowments $w^i$ or $\theta_j^i$.
		
		\subsection{Aggregation}
		
		The aggregation of Marshallian gives the aggregate demand
		\[ x(p) = \sum_i x^i(p)\]
		
		Graphically, it is the sum of individual demand curve
		\begin{figure}[h]
			\center
			\includegraphics[width=0.45\textwidth]{AD.png}
		\end{figure}
		
		
		Now the aggregate supply function for firms is 
		\[ q(p) = \sum_j q^j(p_j)\]
		graphically, this is 
		\begin{figure}[h]
			\center
			\includegraphics[width=0.45\textwidth]{AS.png}
		\end{figure}	
		
		The equilibrium is therefore characterized by a price $p^\ast$ such that 
		\[x(p^\ast) = q(p^\ast)\]
		
		Since this is the aggregation of individual supply and demand, given the price $p^\ast$, we can back out the individual demand and supply.
		
		\subsubsection{Inverse of supply and demand}
		
		The inverse of aggregate supply is 
		\[q^{-1}(\tilde{q})=\tilde{p}\]
		which simply says given the price $\tilde{p}$, the quantity supplied is $\tilde{q}$
		
		Similarly, the inverse aggregate demand is given by 
		\[x^{-1}(\tilde{x})=\tilde{p}\]
		which gives the quantity demanded given the price. 
		Since the demand curve is the individual marginal benefit, or willingness to pay, the inverse aggregate demand is the marginal social benefit.
		\[P(\tilde{x})=x^{-1}(\tilde{x})\]
		
		In equilibrium we have
		\[P(\cdot) = p^\ast = C'(\cdot)\]
		
		\subsubsection{Tax example}
		Suppose we now have a sales tax on good $x$ such that consumer pays $p+t$ for each unit of $x$ while the firm receives $p$.
		
		The consumer's problem is the same as before but with the addition of tax
		\begin{align*}
			\max_{x^i, m^i} m^i + \phi^i(x^i)
		\end{align*}
		subject to 
		\begin{align*}
			m^i + (p^\ast+t) x^i \leq \omega^i + \sum_j \theta^i_j \pi^{j\ast}
		\end{align*}
		
		Thus, the optimal solution is merely
		\[ \hat{x}^i(p) = x^i (p+t)\]
		Thus, the equilibrium is 
		\[ x(p^\ast +t ) = q(p^\ast)\]
		Since the tax has changed the equilibrium price, we have
		\[ x(p^\ast(t) +t ) = q(p^\ast(t))\]
		Assuming $x(\cdot)$ and $q(\cdot)$ are differentiable, take derivative wrt $t$ gives
		\[ x' (p'+1) = q' p^{\ast \prime}\]
		Rearrange gives
		\[ p^{\ast \prime} (t) = \frac{x'(p^\ast +t)}{q'(p^\ast) - x'(p^\ast +t)}\]
		
		Since $x'(\cdot) <0$, $q'(\cdot)\geq 0$
		\[ -1\leq p^{\ast \prime} (t) <0\] 
		i.e. the price received by producer falls.
		
		Consumer pays $p^\ast(t) +t$, which means 
		\[ \frac{\partial [p^\ast(t) +t]}{\partial t} = p^{\ast \prime}(t) +t\]
		where $0 \leq  p^{\ast \prime}(t) +t <1$, so price for consumer weakly rises.
		
		\begin{itemize}
    	\item \( q'(\cdot) \rightarrow \infty \) (elastic supply) implies \( p^{*'}(t) \rightarrow 0 \).
    	\item \( q'(\cdot) = 0 \) (inelastic supply) implies \( p^{*'}(t) = -1 \).
    	\item \( x'(\cdot) \rightarrow -\infty \) (elastic demand) implies \( p^{*'}(t) \rightarrow -1 \).
    	\item \( x'(\cdot) \rightarrow 0 \) (inelastic demand) implies \( p^{*'}(t) \rightarrow 0 \).
		\end{itemize}
		
		\subsection{First welfare theorem}
				
		With quasi-linear utility, any Pareto optimal outcome must maximize the sum of individual utilities, i.e. any Pareto optimal allocation must solve.
		
		\[
		\bar{U}^* = \max_{(x_1, \dots, x_I, q_1, \dots, q_J, m_1, \dots, m_I)} \sum_i u^i(x^i, m^i)
		\]
		subject to feasibility
		\[
		\sum_i x^i \leq \sum_j q^j
		\]
		and
		\[
		\sum_i m^{i*} + \sum_j c^j(q^{j*}) \leq w
		\]
		
		Since we are dealing with utility maximization, both constraint must hold with equality, so the problem becomes 
		\[
		\bar{U}^* = \max_{(x_1, \dots, x_I, q_1, \dots, q_I)} \underbrace{\sum_i \phi^i(x^i) - \sum_j c^j(q^j)}_{\text{Marshallian surplus}} + w
		\]
		subject to
		\[
		\sum_i x^i = \sum_j q^j
		\]
		Where Marshallian aggregate surplus is exactly the utility incurred from $x$ minus of cost of producing it.
		
		Thus, the necessary and sufficient condition for Pareto optimal (given convexity assumptions) are 
		\begin{enumerate}
    	\item \(\lambda \leq c^{j'}(q^{j*})\) for all \(j\)
   		 with equality for positive output.
    	\item \(\phi^{i''}(x^{i*}) \leq \lambda\) for all \(i\)
   		with equality for positive consumption of \(x\).
    	\item \(\sum_i x^{i*} = \sum_j q^{j*}\)
		\end{enumerate}
		which is the same as competitive equilibrium, the we have proved the first welfare theorem implicitly.
		
		\subsection{Second welfare theorem}
		
		The \textbf{Second Welfare Theorem} says Pareto optimal allocation is achievable can be achievable via competitive equilibrium with transfers, more formally:
		
		If there exist a feasible allocation such that 
		\[\sum_i u^i (m^i, x^i) = \bar{U}\]
		Then there exists a feasible allocation that yields any set of individual utility level such that 
		\[\sum_i u^i (\hat{m}^i, \hat{x}^i) = \bar{U}\]
		which can be achieved by holding consumption and production of $x$ constant and redistribute the numeraire good $m$. This can be achieved by redistribution of the endowment of numeraire good $\omega^i$ via transfers $(T^1, \ldots, T^l$ such that $\sum_i T^i = 0$, where an competitive equilibrium with endowments \((\omega^1 + T^1, \dots, \omega^I + T^I)\) yields the utilities \((u^{1*}, \dots, u^{I*})\).
		
		Further, this implies any allocation that cannot be implemented with a competitive equilibrium with transfers cannot be Pareto optimal.
		
		\subsection{Distortionary policies}
		In any competitive equilibrium with interior solution, we have 
		\begin{enumerate}
    	\item For firms: \( p^* = MC^j \)
   		\item For consumers: \( p^* = MB^i \)
    	\item Which implies \( MC^j = MB^i \)
		\end{enumerate}
		
		If we have a tax such that firms receive $p(1-\tau)$ for output good, but consumers pay $p$. Now the condition becomes 
		\begin{enumerate}
    	\item For firms: \( (1-\tau)p^* = MC^j \)
   		\item For consumers: \( p^* = MB^i \)
    	\item Which implies \( MC^j \neq MB^i \)
		\end{enumerate}
		
		Or if a firm has market power so that they know that their output will have an effect on equilibrium price, meaning now the firm solves 
		\[ \max_y p(q)q -c(q)\]
		Now the condition becomes
		\begin{enumerate}
    	\item For firms: \(\frac{\partial p(q^*)}{\partial q} q^* + p^* = MC^j\)
    	\item For consumers: \( p^* = MB^i \)
    	\item Which implies \( MC^j \neq MB^i \)\\[4pt]
    	if \(\frac{\partial p(q^*)}{\partial q} \neq 0\)
		\end{enumerate}
		
		An allocation is optimal must maximize aggregate surplus 
		\[S(x^1, \dots, x^I, q^1, \dots, q^J) = \sum_i \phi^i(x^i) - \sum_j c^j(q^j)\]
		Let $x=\sum_i x^i$ be the aggregate consumption of $x$, we must have $\phi^{i \prime}(x^i) = P(x)$, as the marginal benefit of consumption must equal to the marginal cost of the good. 
		
		Let $q=\sum_j q^j$ be the aggregate production of good. The marginal cost must equalized across producers, since marginal cost equals price. Thus we have \(c^{j\prime} = C'(q)\) for all $j$.
		
		\subsection{Surplus}
						
		Assume a change in quantities consumed and produced that satisfies 
		\[ \sum_i dx^i = \sum_j dq^j = dx\]
		
		Take total differential gives the change in surplus
		\[
		dS = \sum_i \phi^{i'}(x^i) dx^i - \sum_j c^{j'}(q^j) dq^j
		\]
		Replacing with $\phi^{i \prime}(x^i) = P(x)$ and \(c^{j\prime} = C'(q)\) gives
		\[
		dS = P(x) \sum_i dx^i - C'(q) \sum_j dq^j
		\]
		Plugging in $dx$
		\[
		dS = [P(x) - C'(x)] dx
		\]
		
		\begin{figure}[h]
			\center
			\includegraphics[width=0.45\textwidth]{Ds.png}
		\end{figure}
		
		The total surplus can then be written as 
		\[
		S(x) = S_0 + \int_0^x [P(s) - C'(s)] ds
		\]
		which is maximized at the competitive equilibrium level $x^\ast$ where 
		\[ P(x^\ast) = C'(x^\ast)\]
		
		\begin{figure}[h]
			\center
			\includegraphics[width=0.45\textwidth]{TS.png}
		\end{figure}
		
		\subsubsection{Welfare effect of distortionary tax}
		Let \((x^{1*}(t), \dots, x^{I*}(t), q^{1*}(t), \dots, q^{J*}(t))\) denote equilibrium quantities and \(p^*(t)\) given tax rate \(t\).
		
		For all consumers we have 
		\[
		\phi^{i'}(x^{i*}(t)) = p^*(t) + t
		\]
		
		For all producers we have 
		\[
		c^{j'}(q^{j*}(t)) = p^*(t)
		\]
		
		The effect on welfare of change
		\[
		S(x^*(t)) - S(x^*(0)) = \int_{x^*(0)}^{x^*(t)} [P(s) - C'(s)] ds
		\]
		Since we know $x^\ast(t) < x^\ast (0)$, so we know $[P(x) - C'(x)]\geq 0$ for $x < x^\ast(0)$. So there is a positive decrease in surplus, which is called deadweight loss. 
		
		\begin{figure}[h]
			\center
			\includegraphics[width=0.45\textwidth]{DWL.png}
		\end{figure}
		
		This deadweight loss can be approximated with Harberger triangle, and $x^\ast (t) - x^\ast (0)$ is a sufficient statistic for the deadweight loss. \\
		
		\hrule\
		
		\textbf{Aggregate consumer surplus} is the gross consumer benefit from consumption minus the total expenditure.
		\[
		CS(\hat{p}) = \sum_i \phi^i(x^i(\hat{p})) - \hat{p} x(\hat{p})
		\]
		where $\hat{p}$ is the effective price faced by consumer. Since
		\[
		\sum_i \phi^{i'}(x^i) = \int_0^x P(s) ds
		\]
		gives
		\[
		CS(\hat{p}) = \int_0^{x(\hat{p})} [P(s) - \hat{p}] ds = \int_{\hat{p}}^{\infty} x(s)ds
		\]
		
		Now the aggregate producer surplus is 
		\[
		\Pi(\hat{p}) = \hat{p} q(\hat{p}) - \sum_j c^j(q^j(\hat{p}))
		\]
		If we again denote $hat{p}$ as the price faced by producer
		\[
		\Pi(\hat{p}) = \Pi_0 + \int_0^{q(\hat{p})} [\hat{p} - C'(s)] ds = \Pi_0 + \int_0^{\hat{p}} q(s) ds
		\]
		
		Therefore, change in consumer surplus from imposition of tax:
		\[
		\Pi(p^*(t)) - \Pi(p^*(0)) = -\int_{p^*(t)}^{p^*(0)} q(s) ds
		\]
		
		\section{Externality}
		\subsection{Simple example}
		Externalities are production and utility directly affected by other agents in the economy.
		
		With externalities, competitive equilibrium is not Pareto optimal, as marginal cost no longer equals to marginal benefit when agent choose with externality.
		
		We use a very simple example to illustrate this idea. Take the quasi-linear utility function for 2 agents.
		\[u^i (h,m^i) = \phi^i(h^1) +m^i\]
		where we assume $\phi^i$ is twice differentiable and strictly concave.
		
		Now, with externality, $h^1$ chosen by consumer 1 but affects consumer 2's utility, i.e. \(\frac{\partial \phi^2}{h^1}\neq0\).
		In this setting $\phi^1$ measures the benefit of $h^1$ to consumer 1, while $\phi^2$ measures the externality of $h^1$ consumed by consumer 1. 
		
		Without any restriction, consumer 1 solves 
		\[ \max_h \phi^1 (h^1) \implies \phi^{\prime 1}(h^\ast)=0\]
		given interior solution $h^\ast>0$
		
		Now, a Pareto optimal allocation $h^o$ must satisfy 
		\[ \max_h \phi^1 (h^1) + \phi^2(h^1)\]
		given interior solution $h^o>0$, this is 
		\[\phi^{\prime 1}(h^o) + \phi^{\prime 2}(h^o)=0\]
		 
		\begin{itemize}
    	\item Negative externality:
    	\[
    	\phi^{\prime2}(h^1) < 0 \implies h^* > h^0.
    	\]
    	\item Positive externality:
    	\[
    	\phi^{\prime2}(h^1) > 0 \implies h^* < h^0.
    	\]
		\end{itemize}

		\subsection{Solutions to externality}
		Solutions to externality includes 
		\begin{enumerate}
			\item Quotas
			\item Taxes
			\item Property rights
		\end{enumerate}
		
		\subsubsection{Quota}
		Quota basically works if the government knows $h^o$, it can simply mandate that consumer 1 must produce $h^o$.
		
		\subsubsection{Pigouvian Taxes}
		Assume we have a negative externality $\phi^{\prime2}(h^1) < 0$
		We let consumer 1 pay a tax of $t_h$ per unit of $h$, so his maximization problem is now
		\[ \max_h \phi^1 (h^1) - t_h h\]
		and the FOC condition is 
		\[ \phi^{\prime 1} (h^1) - t_h =0\]
		Now if we institute a tax that is just $t_h = -\phi^{\prime 2}(h^o)$, then we can recover the sufficient condition for Pareto optimal
		\[\phi^{\prime 1}(h^o) + \phi^{\prime 2}(h^o)=0\]
		which is satisfied when $h= h^o$, i.e. optimality restoring tax equals to marginal externality at $h^o$.
		
		\begin{figure}[h]
			\center
			\includegraphics[width=0.45\textwidth]{Op_tax.png}
		\end{figure}
		
		Alternatively, we are essentially instituting a tax $t = -\phi^2(h^1)$, since $\phi^2(h^1)$ is the negative externality, $-\phi^2(h^1)$ is positive, and we just levy the tax that is just equal to the amount of externality faced by consumer 2, which changes the problem of consumer 1 to 
		\[ \max_h \phi^1 (h^1) +\phi^2(h^1)\]
		which is exactly the problem for Pareto optimality.
		\vspace{4pt}
		
		\hrule\
		
		We can also subsidies the reduction of $h$, i.e. now consumer 1 receives a subsidy of $s_h$ for every unit of $h$ they produce below some amount $\bar h > h^o$. Now the maximization problem of consumer 1 is 
		\[ \max_h u^1 (h,m^1)=\phi^1 (h) + s_h(\bar h - h)\]
		FOC is then 
		\[ \phi^{\prime 1}h - s_h =0\]
		Again, if we set $s_h = -\phi^{\prime 2}(h^o)$, we recover the condition for Pareto optimality.
		
		Both method requires fair amount of information. 
		
		\subsubsection{Property right}

		Suppose property rights are well defined and enforceable and a
		competitive market for the right to produce externalities
		exists.
		
		We assume the consumer 2 has property rights to the backyard.
		If he chooses, he can sell the right at $p_h$ for the power plant to
		“use” area in his/her backyard. For now we assume both consumer 1 and 2 are price taker, but this can be relaxed.
		
		The demand $h^1$ from consumer 1 must equal to $h^2$, the supply from consumer 2, in equilibrium.
		
		Now, consumer 1's problem is 
		\[
		\max_{h^1} \phi^1(h^1) - p_h h^1
		\]	
		Given interior solution, the FOC is 	
		\[
		\phi'^1(h^1) = p_h
		\]	
		Consumer 2's problem is 
		\[
		\max_{h^2} \phi^2(h^2) + p_h h^2
		\]
		Given interior solution, the FOC is 
		\[
		\phi'^2(h^2) = -p_h
		\]
		
		In equilibrium, we have $h^1 = h^2 =h^o$, which implies 
		\[\phi'^1(h^1) = - \phi'^2(h^2) = p_h\]
		
		\hrule\
		
		Similar to the relationship between tax and subsidy, with property right, consumer 1 can sell the right to reducing pollution $h^1 < \bar{h}$ to consumer 2. So consumer 1's problem is then
		\[
		\max_{h^1} \phi^1(h^1) + p_h (\bar{h} - h^1)
		\]
		FOC for interior solution is then
		\[
		\phi'^1(h^1) = p_h
		\]
		Consumer 2 then solves
		\[
		\max_{R} \phi^2 (\bar{h} - R) - p_h R
		\]
		where $R$ is the amount of reductions purchased by consumer 2. We can rewrite this as 
		\[
		\max_{h^2} \phi^2 (h^2) - p_h (\bar{h} - h^2)
		\]
		given interior solution, we yield a FOC of 
		\[
		\phi'^2(h^2) = -p_h
		\]
		In equilibrium, we have $h^1 = h^2 =h^o$, which implies 
		\begin{align*}
			\phi'^1(h^1) &= - \phi'^2(h^2) = p_h\\
			\implies \phi'^1(h^o) &+ \phi'^2(h^o)=0
		\end{align*}
		Again, we recover the condition for Pareto optimality
		
		The optimal level of \( h \) does not depend on who holds property rights, but different property rights yield different ex-post utilities. If consumer 2 has property rights, they receive \( p^{**} h^o \) from consumer 1, whereas if consumer 1 has property rights, they receive \( p^{**} (\bar{h} - h^o) \) from consumer 2. This illustrates externalities as a case of ``missing markets''—if markets existed for these goods, the First Welfare Theorem would hold. Thus, well-defined and enforceable property rights are crucial, but challenges arise when \( h \) is difficult to measure.
		
		\subsection{Multilateral externality}
		Now we discuss some more general casts of externality that involves multiple parties. There are 2 types of multilateral externalities. 
		\begin{itemize}
			\item Rivalrous externality (private, depletable) - experience of externality by one agent reduced the amount felt by other agents
			\item Non-rivalrous externality (public, nondepletable) - experience of externality does not affect amount felt by other agents
		\end{itemize}
		
		\subsubsection{Rivalrous externality}
		Assume there are $I$ consumers who produce the externality and $K$ consumers that experience the externality.
		
		The Pareto optimal allocation satisfies 
		\begin{align*}
    	\max_{h^1, \dots, h^J, \tilde{h}^1, \dots, \tilde{h}^K} &\quad 
    	\sum_{i \in I} \phi^i(h^i) + \sum_{k \in K} \tilde{\phi}^k (\tilde{h}^k) \\[10pt]
    	&\text{s.t.} \quad \sum_{i \in I} h^i = \sum_{k \in K} \tilde{h}^k
		\end{align*}
		where $\phi^i$ is the utility from producing the externality, and $\tilde{\phi}^k$ is the utility from experiencing the externality. 
		
		\subsection{Non-rivalrous externality}
		The Pareto optimal allocation satisfies 
		\begin{align*}
    	\max_{h^1, \dots, h^J, \tilde{h}^1, \dots, \tilde{h}^K} &\quad 
    	\sum_{i \in I} \phi^i(h^i) + \sum_{k \in K} \tilde{\phi}^k (\tilde{h}) \\[10pt]
    	&\text{s.t.} \quad \sum_{i \in I} h^i =  \tilde{h}
		\end{align*}
		 
		 Plugging in gives
		 \[\max_{h^1, \dots, h^J, \tilde{h}^1, \dots, \tilde{h}^K} \quad \sum_{i \in I} \phi^i(h^i) + \sum_{k \in K} \tilde{\phi}^k (\sum_{i \in I} h^i)\]
		 which gives the FOC
		 \[\phi^{i\prime}(h^{io})\leq - \sum_{k\in K} \tilde{\phi}^{k \prime} \left(\sum_{i \in I} h^{io}\right)\]
		 
		 Again, we can set the tax or subsidy to be
		 \[ \tau_h = - \sum_{k\in K} \tilde{\phi}^{k \prime} \left(\sum_{i \in I} h^{io}\right)\]
		 So the externality producers solve
		 \[ \max_{h^i} \phi^i (h^i) - \tau_h h^i\]
		 which yields the FOC
		 \[\phi^{i \prime} (h^{io}) \leq \tau_h\]
		 Using the optimal tax recovers the Pareto optimality
		 \[\phi^{i\prime}(h^{io})\leq - \sum_{k\in K} \tilde{\phi}^{k \prime} \left(\sum_{i \in I} h^{io}\right)\]
		 
		 \subsection{Some comments}
		 In general, solving externalities requires a lot of information.
		 \begin{itemize}
		 	\item Quotes - need to know $h^{io}$ for all producers of externality
		 	\item Taxes/subsidy - need to know marginal social cost of externality at optimum
		 	\item Cap and trade - requires only $\sum_i h^{io}$, as it put quotea on $\sum_{i\in I} h^i$ and distribute permits to produce externality while prices are determined endogenously.
		 \end{itemize}
		
		
		
		
		
		\section{Public Goods}	
		
		A public good must satisfy 
		\begin{itemize}
			\item Non-rivalrous - nondepletable 
			\item Non-excludable
		\end{itemize}	
		The discussion follow on focuses on ``pure public good'', where it is totally nondepletable. But could have congestion effect.
		
		\subsection{Setup}
		$I$ consumers have quasi-linear utility function over public good $x$ and money $m^i$.
		\[ u^i(x, m^i)= \phi^i(x) +m^i\]
		where $\phi^i$ is increasing and concave, and the cost of producing public good is  \(c(q) \).
		
		So the Pareto optimal allocation must satisfy
		\begin{align*}
			&\max_{x,q,m^i} \sum_i^I u^i (x,m^i) = \sum_i u^i(x) +\sum_i m^i \\
			&s.t. \quad x \leq q\\
			&\quad c(q) +\sum_i m^i \leq \omega
		\end{align*}
		
		Plugging in the constraints gives 
		\[ \max_{q} \sum_i^I \phi^i(q)-c(q)\]
		which is maximizing Marshallian surplus. 
		
		\subsection{Private provision of public goods}
		Consumers choose how many public goods to purchase, but they benefit from the total amount of public goods, and they are non-excludable and non-rivalrous. So the individual maximization problem is 
		\[ \max_{x^i} \phi^i(x^i + \sum_{k\neq i} x^{k \ast})-p^\ast x^i\]
		Take derivate wrt to $x^i$ gives 
		\[\phi^{\prime i}(x^i + \sum_{k\neq i} x^{k \ast})=p^\ast\]
		
		\hrule\
		
		Now the firm sets 
		\[p^\ast = c'(q)\]
		i.e. marginal benefit equals marginal cost, so 
		\[\phi^{\prime i}(x^i + \sum_{k\neq i} x^{k \ast}) = c'(q)\]
		So each individual demand public goods until marginal benefit equals to marginal cost, if we sum up the LHS:
		\[\sum_i \phi^{\prime i}(x^i + \sum_{k\neq i} x^{k \ast}) =\sum_i \phi^{\prime i}(q)> c'(q)\]
		But the optimal level should be
		\[\sum_i \phi^{\prime i}(q)= c'(q)\]
		Since $\phi^i$ is concave, $q^\ast<q^o$
		
		This is exactly the free-rider problem, consumers incentivized to enjoy benefits of
		public good but under-provide. Can be remedied via quantity-based intervention (direct government intervention) or price based intervention (subsidies)
		
		
		\subsection{Hedonics}
		This discussion is based on \emph{Rosen 1974}. 
		We consider a good that can be described by a vector of its characteristics 
		\[\mathbf{C} = (c_1, c_2, \ldots, c_n)\]
		and the market price is a function of all the characteristics
		\[ P(\mathbf{C}) = P(c_j, \mathbf{C}_{-j})\]
		if we want to separate the effect of $c_j$, where $\mathbf{C}_{-j}$ is all the other characteristics except $c_j$.
		
		Now suppose a household wants to consume this good with characteristic vector $\mathbf{C}$, then his utility function is 		
		\[
		\max_{\mathbf C, m} u^i (\mathbf C, m)
		\]
		subject to \(P(\mathbf C) + m = I\).\\[6pt]
		Plugging in the constraint
		\[
		\max_{\mathbf{C}} u^i (\mathbf{C}, I - P(\mathbf{C}))
		\]
		If we hold $\mathbf{C_{-j}}$ constant:
		\[
		\max_{c_j} u^i (c_j, \mathbf{C_{-j}}, I - P(\mathbf{C}))
		\]
		 
		\begin{figure}[H]
			\center
			\includegraphics[width= 0.5\textwidth]{hedonic.png}
		\end{figure}
		An example is given here in house prices and air quality. Different consumers will choose different tangency points along Hedonic price schedule, which gives the willingness to pay for that household. 
		
		\subsection{Optimal Need Based Financial Aid}
		We denote $s$ to be the financial aid schedule, 
		
		$E_i(s_i)$ to be  share of students going to college at income level $i$, 
		
		$TR(s)$ to be the total tax revenue given schedule $s$ and 
		
		$cost(s)= \sum_i E_i (s_i)s_i$ to be the total cost of financial aid 
		
		If we maximize the social welfare 
		\[\max_{s} W(s)\]
		subject to \( TR(s) \geq cost(s)\), Lagrangian gives
		\[ W(s_i) - \lambda [TR(s_i)- cost(s_i)]\]
		Take FOC gives 
		\[ \frac{\partial E_i}{\partial s_i} \Delta T_i^E - E_i (1-W_i^E)=0\]
		
		
		
\end{document}