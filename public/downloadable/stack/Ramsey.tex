%! Author = herbertxin
%! Date = 11/16/24

\documentclass[twocolumn, fleqn]{article}
\usepackage[margin=1in]{geometry}
\usepackage{fancyhdr}
\usepackage{graphicx}
\usepackage{amsmath}
\usepackage{hyperref}
\usepackage{enumitem}
\usepackage{amssymb}
\usepackage{tcolorbox}



% Adjust column separation and add vertical rule
\setlength{\columnsep}{25pt} % Increase space between columns
\setlength{\columnseprule}{0.6pt}
\setlength{\jot}{6pt} % Set the space between equation lines to 6pt
\setlength{\parskip}{3pt}

% Header and Footer Settings
\pagestyle{fancy}
\fancyhf{}
\lhead{Core Macro I}
\chead{Ramsey Model}
\rhead{Page \thepage}
\renewcommand{\headrulewidth}{0.4pt}
\newtcolorbox{note}[1][]{
	enhanced,
	colback=white,            % Background color of the box content
	colframe=black,           % Border color
	coltitle=white,           % Title text color
	colbacktitle=black,       % Background color of the title area
	fonttitle=\bfseries,      % Bold font for the title
	boxrule=1pt,              % Border thickness
	title=SIDENOTE,           % Title of the box
	sharp corners,            % Square corners
	before skip=10pt,         % Space before the box
	after skip=10pt,          % Space after the box
	left=5pt,                 % Left padding
	right=5pt,                % Right padding
	top=5pt,                  % Top padding
	bottom=5pt,               % Bottom padding
	width=\columnwidth,       % Box width matches column width
	#1                        % Allows for optional parameters
}

\begin{document}

	\title{Ramsey Model}
	\author{Herbert W. Xin}
	\date{\today}
	\maketitle

	\tableofcontents
	\thispagestyle{fancy}

	\section{Assumptions}\label{sec:assumptions}
		\begin{enumerate}
			\item Household are identical and live forever, closed economy.
			\item Unique final goods produced by a large number of identical firms $Y = F(K,AL)$, which follows the
			neoclassical assumptions:
				\begin{enumerate}
					\item $F$ is CRS in $K \& L$
					\item $F_K, F_L>0, F_{KK}, F_{LL}<0, F_{KL}>0$
					\item Inada conditions
				\end{enumerate}
			\item Markets are perfectly competitive, and firms solve the profit maximization problem
				\begin{enumerate}
					\item $\frac{\partial F}{\partial L}=W \implies W_t = A_t [f(k_t)-f'(k_t)k_t]$
					\item $\frac{\partial F}{\partial K}=Q \implies Q_t = f'(k_t), r_t = Q_t -\delta=f'(k_t)-\delta$
				\end{enumerate}
			\item Technology and labor grow at a constant rate
			    \begin{enumerate}
					\item $A_{t+1} = (1+g)A_t$
					\item $L_{t+1} = (1+n)L_t$
			    \end{enumerate}
			\item There is one representative household that work for $t = 0,1,2,\ldots,\infty$, which contains $L_t$ members each having 1 unit of time that they work for (no labor-leisure choice). So, $L_t$ is the labor supply of the household.
			\item \textbf{Notation:} $C_t$ in the Ramsey model stands for consumption per labor, so $c_t = \frac{C_t}{A_t}$ is the
			consumption per effective labor. $k_t = \frac{K_t}{A_t L_t}$ is the capital per effective labor.
		\end{enumerate}

		\begin{note}
			The economy has no externalities, market frictions, policy distortions, or the ``double infinity''
			problem of OLG.
			The resulting competitive equilibrium in Pareto Optimal, and we can solve for it from either the \textbf{social
			planner} problem or the market-based \textbf{decentralized} problem.
			The straight forward way of solving these problems is using the Lagrangian method, but with more
			complicated problem, dynamic programming is a more elegant way.
		\end{note}

	\section{Finite horizon problem and transversality condition}
		\label{sec:finite-time-problem-and-transversality-condition}
		This section demonstrates an important assumption of the Ramsey model, or any infinite time horizon model
		hereafter, which is the transversality condition.

		The transversality condition ensures the UMP has a defined solution, which is achieved by placing
		restrictions on household's budget constraint.
		If the budget constraint is not bounded, then UMP cannot be solved nor it has an economic meaning.

		The strip-down version of the transversality condition is the No-Ponzi Game (NPG) condition, which prohibits
		household from borrowing infinitely from the future.
		If the NPG is violated, then the household essentially has infinite wealth and UMP cannot be solved.

		\subsection{Setup}\label{subsec:setup}
			Suppose the economy solves a finite horizon decentralized problem
			\[\max \sum_{t=0}^{T} \beta^t u(C_t)L_t \]

			s.t. $C_t L_t + K_{t+1}= W_t L_t + Q_t K_t + (1-\delta)K_t, \forall t \in [0,T]$.

			Note we can rewrite $Q_t K_t + (1-\delta)K_t = (1+Q_t -\delta)K_t = R_t K_t$.

			Now the NPG condition, \[\frac{K_{T+1}}{\bar{R_t}}\geq 0, \forall k_0 >0\]
			where $\bar{R}_T = \prod_{s=1}^T R_s$

			The NPG condition ensures household cannot hold debt at the terminal period, if they do, the budget
			constraint is not bounded.

		\subsection{NPG to budget set}\label{subsec:npg-to-budget-set}
			Given the budget constraint, rearrange gives
			\begin{gather*}
			    K_t R_t =(C_t - W_t)L_t+K_{t+1}\\
			    \implies K_t =\frac{(C_t - W_t)L_t}{R_t}+\frac{K_{t+1}}{R_t}\\
			\end{gather*}

			Iterate from $t=0$:
			\begin{gather*}
				(C_0-W_0)L_0 = R_0 K_0 - K_1 \\
				= R_0 K_0 - \frac{(C_1 - W_1)L_1}{R_1} - \frac{K_2}{R_1}\\
				(C_0 -W_0)L_0 + \frac{(C_1-W_1)L_1}{R_1} = R_{0} K_0 - \frac{K_2}{R_1}
			\end{gather*}

			Interate infinitely gives
			\[
				\sum_{t=0}^{T}\frac{(C_t-W_t)L_t}{\bar{R}_t} = R_0 K_0 - \frac{K_{T+1}}{R_T}
			\]

			Then the combination of NPG condition and budget constraints gives
			\[
				0 \leq \frac{K_{T+1}}{R_T} \leq R_0 K_0
			\]

			This is equivalent of saying
			\[
				\sum_{t=0}^{T} \frac{(C_t-W_t)L_t}{\bar{R}_t} \leq R_0 K_0
			\]

			Rearrange gives
			\[
				\sum_{t=0}^{T} \frac{C_t L_t}{\bar{R}_t} \leq \sum_{t=0}^{T} \frac{W_t L_t}{\bar{R}_t} R_0 K_0
			\]

			The equation above basically says the PV of lifetime consumption needs to be less than equal to the PV
			of lifetime labor income plus non-labor income (financial wealth)

	\subsection{Optimal Choice}\label{subsec:optimal-choice}
		Now we set up the lagrangian
		\[
			\mathcal{L} = \sum_{t=0}^{T} \beta^t u(C_t)L_t + \sum_{t=0}^{T} \lambda_t [W_t L_t + R_t K_t - C_t L_t - K_{t+1}] + \mu \frac{K_{T+1}}{\bar{R}_T}
		\]

		F.O.C gives
		\begin{align*}
			\beta^t u'(C_t)L_t - \lambda_t L_t &= 0, \ \forall t \in [0,T] \tag{2.1}\\
			- \lambda_t + \lambda_{t+1} R_{t+1} &= 0, \ \forall t \in [0, T-1] \tag{2.2}\\
			- \lambda_t + \frac{\mu}{\bar{R}_T} &= 0, \ \text{ for } t =T \tag{2.3}\\
			\mu \frac{K_{T+1}}{\bar{R}_T} &= 0 \tag{C.S.}
		\end{align*}

		Combine (2.1) and (2.2) gives
		\begin{align*}
			\beta^t u'(C_t) &= \lambda_t \\
			\lambda_t &= \lambda_{t+1} R_{t+1}
		\end{align*}

		This gives the Euler equation
		\[u'(C_t) = \beta R_{t+1} u'(C_{t+1}) \tag{Euler Equation}\]

		Combine (2.3) and complementary slackness gives
		\begin{align*}
			\mu \frac{K_{T+1}}{\bar{R}_T} &= \lambda_T K_{T+1}\\
			&= \beta^t u'(C_t)K_{T+1} = 0 \tag{Transversality}
		\end{align*}
		where $\beta^t u'(C_t)$ means the shadow value of terminal capital stock or wealth equals $\mu$ from
			consumption, which is satisfied with $K_{T+1}=0$ as long as $u'(C)>0, \forall C \geq 0$

		As $T \rightarrow \infty$, NPG implies \[\lim_{T\rightarrow \infty} \frac{K_{T+1}}{\bar{R}_T}\geq 0\]
			and TVC implies \[\lim_{T\rightarrow \infty} \beta^T u'(C_T)K_{T+1} = 0\]

	\section{Decentralized Optimization}\label{sec:decentralized-optimization}
		There are two ways to solve the Ramsey model, the first one is to solve households utility maximization
		problem, i.e. the decentralized optimization problem.
		The second way is to solve the social planner's optimization problem, which should yield the same result as
		the decentralized version.

		In the decentralized problem, household optimization FOC gives the Euler equation, combined with the TVC and
		budget constraint allows us to solve for the equlibrium.
		\begin{align}
			u'(C_t) &= \beta R_{t+1}u'(C_{t+1})\\
			\lim_{t \rightarrow \infty} \beta^t &u'(C_t)K_{t+1} = 0\\
			C_t L_t + K_{t+1} &= W_t L_t + Q_t K_t + (1-\delta)K_t
		\end{align}

		The household take factor prices as given
		\begin{align}
			W_t &= A_t [f(k_t)-f'(k_t)k_t]\\
			Q_t &= f'(k_t), r_t = Q_t -\delta=f'(k_t)-\delta
		\end{align}

		To obtain an analytical solution, we assume a CES utility function
		\begin{equation}\label{eq:equation}
			u(C_t) &= \frac{C_t^{1-\sigma}}{1-\sigma}, \sigma >0
		\end{equation}

		Substitute this into (1) gives
		\begin{align*}
			C_t^{-\sigma} &= \beta R_{t+1} C_{t+1}^{-\sigma}\\
			(c_t A_t)^{-\sigma} &= \beta R_{t+1}(c_{t+1}A_{t+1})^{-\sigma}\\
			c_t^{-\sigma} &= \frac{\beta R_{t+1}}{(1+g)^\sigma}c_{t+1}^{-\sigma} \tag{8a}\\
			\text{As } A_{t+1}&= (1+g)A_t
		\end{align*}

		This gives the k locus, as when $c_t = c_{t+1}$, it depicts the steady state level of capital.

		In general
		\begin{equation}
			u'(c_t) = \frac{\beta R_{t+1}}{(1+g)^\sigma} u'(c_{t+1})
		\end{equation}

		Now we take the budget constraint,
		\begin{align*}
			C_t L_t + K_{t+1} &= W_t L_t + Q_t K_t + (1-\delta)K_t\\
			c_t + \frac{K_{t+1}}{A_t L_t} &= \frac{W_t}{A_t} + Q_t \frac{K_t}{A_t L_t} + (1-\delta)\frac{K_t}{A_t L_t}\\
			c_t + (1+z)k_{t+1} &= w_t + Q_t k_t + (1-\delta) \frac{K_t}{A_t L_t}\\
			c_t + (1+z)k_{t+1} &=f(k_t) - k_t f'(k_t) + k_t f'(k_t)\\
			&+(1-\delta)k_t\\
			(1+z)k_{t+1}&= f(k_t) + (1-\delta)k_t - c_t \tag{8b}
		\end{align*}

		Now, CES utility function implies
		\[u'(C_t) = C_t^{-\sigma}=(c_t A_t)^{-\sigma}=\frac{u'(c_t)}{[(1+g)^t A_0]^\sigma}\]

		Substitute this into equation 2 gives
		\begin{gather*}
		    \lim_{t \rightarrow \infty} \beta^t \frac{u'(c_t)}{[(1+g)^t A_0]^\sigma} \underbrace{\left[(1+g)^{t+1}A_0(1+n)^{t+1}L_0 k_{t+1}\right]}_{K_{t+1}} =0\\
		    \lim_{t \rightarrow \infty} \underbrace{\beta^t (1+n)^t (1+g)^{(1-\sigma)t}}_{\theta^t} \underbrace{\left[(A_0^{1-\sigma}L_0(1+g)(1+n)\right]}_{\text{all positive constant}}\\
		    \cdot u'(c_t)k_{t+1} =0\\[4pt]
			\implies \lim_{t \rightarrow \infty} \theta^t u'(c_t)k_{t+1} =0 \tag{9}
		\end{gather*}

		Now, equation 8a and 8b describes a non-linear difference equation system in $(k_t, c_t)$ that decribes the
		dynamic general equilibrium of the Ramsey economy.
		Equation 9 and the initial condition $k_0 \equiv \frac{K_0}{A_0L_0}>0$ serve as the two boundary conditions.

		The dynamics of Ramsey model needs to be analyzed with the asistance of a phase diagram.
		(See ``Stability'' file for a more rigorous discussion on stability of non-linear systems)

		\section{Equilibrium Dynamics}\label{sec:equilibrium-dynamics}
		The two characterizing equations are
		\begin{align}
			c_{t+1}^{\sigma} &= \frac{\beta [1+f'(k_{t+1})-\delta)]}{(1+g)^\sigma}c_{t}^{\sigma} \label{c-locus}\\
			(1+z)k_{t+1} &= f(k_t) + (1-\delta) k_t - c_t \label{k-locus}
		\end{align}

		Now $\Delta k_t \geq 0$ implies
		\begin{align*}
			k_{t+1} &\geq k_{t} \\
			(1+z)k_t &\leq (1+z)k_{t+1} = f(k_t) + (1-\delta) k_t - c_t\\
			c_t &\leq f(k_t) - (z+\delta)k_t
		\end{align*}

		and $\Delta c_t \geq 0$ implies
		\begin{align*}
			c_{t+1} &\geq c_t \\
			c_{t+1}^{\sigma} &\geq c_t^{\sigma}\\
			\frac{\beta [1+f'(k_{t+1})-\delta)]}{(1+g)^\sigma} &\geq 1\\
			f'(k_{t+1}) &\geq \frac{(1+g)^\sigma}{\beta} - (1-\delta) \\
			&\equiv f'(\bar k)
		\end{align*}

		\begin{gather*}
			\implies k_{t+1} \leq \bar{k}\\
			\text{where } \bar k \equiv \frac{(1+g)^\sigma}{\beta} - (1-\delta)\\
			(1+z)k_{t+1} \leq (1+z)\bar{k}\\
			f(k_t)+(1-\delta)k_t - c_t \leq (1+z)\bar{k}\\[4pt]
			\implies c_t \geq f(k_t)+(1-\delta)k_t - (1+z)\bar{k}
		\end{gather*}

		The comparative dynamics based on the two equation are depicted on \eqref{fig:phase}.
		Note, the c-locus is a curve rather than a straight line as in continuous time version.

		\section{Steady State Analysis}\label{sec:steady-state-analysis}
		At steady state $k_{t+1} = k_t = k, c_{t+1}=c_t = c$

		Substitute into \eqref{c-locus} and \eqref{k-locus} gives the ``non-trivial'' pairwise steady state $(\bar{k}, \bar{c})$
		\begin{align}
			f'(\bar{k}) &= \frac{(1+g)^\sigma}{\beta} - (1-\delta)\\
			\bar{c} &= f(\bar{k}) - (z+\delta)\bar{k}
		\end{align}

		\subsection{Golden Rule}\label{subsec:golden-rule}
			Again, golden rule maximizes the steady state consumption per effective labor.

			$\bar{c}$ is maximized of $k_{\text{GR}}: f'(k_{\text{GR}})= z +\delta$

			An natural question is ask is, when is $k_{\text{GR}}>\bar k$, note this implies
			\begin{align*}
				f'(k_{\text{GR}}) &<f'(\bar{k})\\
				z+\delta &< \frac{(1+g)^\sigma}{\beta} - (1-\delta)\\
				(1+z) &< \frac{(1+g)^\sigma}{\beta}\\
				\beta(1+n)(1+g)^{1-\sigma} &<1
			\end{align*}

			Note from equation (9), $\beta(1+n)(1+g)^{1-\sigma}$ is $\theta$, so $k_{\text{GR}}>\bar k$ when $\theta<1$

			$\bar{k}$ is also called the modified Golden Rule capital stock.

		\subsection{The zero steady state}\label{subsec:the-zero-steady-state}

			Another steady state for the Ramsey economy is $(k^\ast, 0)$, however, this steady state violates the
			transversality condition and thus cannot be optimal.

			Note the transversality condition implies
			\[\lim_{t \rightarrow \infty} \theta^t u'(c_t)k_{t+1} =0 \tag{9}\]

			and the Euler equation implies
			\begin{align*}
				u'(c_t) &= \frac{\beta R_{t+1}}{(1+g)^\sigma} u'(c_{t+1})\\
				u'(c_{t+1}) &= \frac{u'(c_t)(1+g)^\sigma}{\beta R_{t+1}}
			\end{align*}

			Thus, we can write $\theta^t u'(c_t)$ as
			\begin{align*}
				\theta^t u'(c_t) &= \theta^t \frac{u'(c_{t-1})(1+g)^{2\sigma}}{\beta R_{t}}\\
				&= \theta^t \frac{u'(c_0)(1+g)^{\sigma t}}{\beta^t(R_t R_{t-1} R_{t-2}\ldots R_1)}
			\end{align*}

			Note $\theta^t \frac{(1+g)^{\sigma t}}{\beta^t}= (1+z)^t$, this means the transversality condition becomes
			\[\lim_{t \rightarrow \infty} u'(c_0) \frac{(1+z)^t}{\bar{R}_t}k_{t+1}=0\]
			However, $\frac{(1+z)^t}{\bar{R}_t}$ goes to infinity while $k_{t+1}$ converges to $k^\ast$, thus, the
			whole term goes to infinity, violating the transversality condition.

			Along the path converging to $(k^\ast, 0)$, at some point $k_t > k_{\text{GR}}$, same as the OLG model,
			when the economy saves more than the golden rule, it is Pareto inefficient.
			This also implies
			\begin{align*}
				f'(k_t)&<f'(k_{\text{GR}})\\
				f'(k_t)-\delta &<f'(k_{\text{GR}}) -\delta\\
				r_t &< z\\
				\implies R_t &< 1+z
			\end{align*}

			\subsection{Analytical example}\label{subsec:analytical-example}
			In many cases, the Ramsey model cannot be solved analytically.
			Here we provide an example that is simply
			enough to obtain an analytical solution to illustrate the idea of saddle path.

			Suppose $u(C)= \ln(C), f(k)=Bk^\alpha, \delta = 1$, solving households UMP gives the Euler equation
			\begin{align*}
				u'(C_t ) &= \frac{\beta}{1+g}u'(C_{t+1})R_{t+1}\\
				\frac{C_{t+1}}{C_t} &= \frac{\beta}{1+g}[\alpha B k_{t+1}^{\alpha-1}]
			\end{align*}

			The budget constraint follows
			\[(1+z)k_{t+1} = Bk_t^\alpha -C_t\]

			Now we guess and verify that saving is a fraction of output
			\begin{align*}
				k_{t+1} &= \eta y_t \\
				&= \eta B k_t^{\alpha}
			\end{align*}
			where $\eta$ is unkown.

			Substitute this into the budget constraint
			\[C_t = Bk_t^\alpha - (1+z)k_{t+1}= B[1-(1+z)\eta]k_t^\alpha\]

			Substitute above into the Euler equation
			\[\frac{B[1-(1+z)\eta]k_{t+1}^\alpha}{B[1-(1+z)\eta]k_{t}^\alpha}= \frac{\beta}{1+g}[\alpha B k_{t+1}^{\alpha -1}]\]

			This implies
			\[k_{t+1}= \frac{\alpha \beta B}{1+g}k_t^\alpha\]
			which exhibits the same form as our guess, and we can infer that
			\[\eta = \frac{\alpha \beta B}{1+g}\]

			Now, the optimal choices for households are
			\begin{align*}
				c_t &= [1-\alpha \beta(1+n)]Bk_t^{\alpha} \equiv g(k_t)\\
				k_{t+1} &= s_t = \frac{\alpha \beta B}{1+g}k_t^\alpha \equiv h(k_t)
			\end{align*}
			where the first equation is exactly the saddle path.


	\section{Comparative Dynamics}\label{sec:comparative-dynamics}
		\subsection{Unexpected permanent increase in $\beta$}\label{subsec:unexpected}
			Suppose $\beta$ increases unexpectedly and permanently at $T$, note that
			\begin{align*}
				\Delta c_t &= 0 \impies c_t = f(k_t) + (1-\delta)k_t - (1+z)\bar{k}\\
				\Delta k_t &= 0 \implies c_t = f(k_t) - (z+\delta)k_t
			\end{align*}

			Since $f'(\bar k)=\frac{(1+g)^\sigma}{\beta} - (1-\delta)$, which decreases as $\beta$ increases, which
			means $\bar k$ increases.

			The exact dynamics are depicted in phase diagram in figure \eqref{fig:ex1}

		\subsection{Unexpected permanent increase in aggregate productivity}
			Suppose $f(k) = Bk^\alpha$, and $B \rightarrow B'$ at $T$, we have
			\begin{align*}
				\Delta c_t &= 0 \impies c_t = Bk_t^\alpha + (1-\delta)k_t - (1+z)\bar{k}\\
				\Delta k_t &= 0 \implies c_t = Bk_t^\alpha - (z+\delta)k_t
			\end{align*}

			When $B$ goes up, we can see that k-locus clearly goes up.
			For c-locus, note again $f'(\bar k)=\frac{(1+g)^\sigma}{\beta} - (1-\delta)$, which implies
			\[\bar{k} = \left[ \frac{\alpha B}{(1+g)^\alpha/\beta - (1-\delta)} \right]^{1/(1-\alpha)}\]

			So $B$ goes up implies $\bar{k}$ goes up.
			However, this obscures the movement of the c-locus.
			To determine the movement of c-locus, we can analyze the movement of $\bar c$
			\begin{align*}
				\bar{c} &= B \bar{k}^{\alpha} - (z+\delta)\bar{k}\\
				\implies \frac{\partial \bar c}{\partial B}&= \bar{k}^{\alpha} + \underbrace{[\alpha B \bar{k}^{\alpha
				-1}-(z+\delta)]}_{f'(\bar{k})-f'(k_{\text{GR}})>0} \frac{\partial \bar{k}}{\partial B}>0
			\end{align*}

			Thus, the c-locus will shift up.
			The phase diagram of this move is outlined in figure \eqref{fig:ex2}.

	\section{Competitive Equilibrium}\label{sec:competitive-equilibrium}
		A competitive equilibrium of the (normalized) Ramsey economy consists of a time path of allocations
		$\{k_t, c_t\}_{t=0}^{\infty}$ and price $\{w_t, r_t\}_{t=0}^{\infty}$, subject to

		(i) the representative household maximizes its utility given $k_0>0$ and taking the time path of prices as
		given, i.e. the conditions below are satisfied

		\begin{align*}
			u'(c_t) &= \frac{\beta}{(1+g)^\sigma}u'(c_{t+1})(1+r_{t+1})\\
			(1+z)k_{t+1} &= f(k_t) + (1-\delta)k_t - c_t \\
			\lim_{t\rightarrow \infty} &\theta^t \frac{k_{t+1}}{\bar{R}_t}=0
		\end{align*}

		(ii) firms maximize profits taking the time path of prices as given, i.e. the conditions below are satisfied.

		\begin{align*}
			w_t &= f(k_t) - k_t f'(k_t)\\
			r_t &= \theta_t -\delta = f'(k_t) -\delta
		\end{align*}

		(iii) factor prices clear all markets

		\begin{note}
			The steady state is a specific outome $k_t = \bar{k}, c_t = \bar{c} \forall t$ and the competitive
			equilibrium is the saddle path that satisfies the path $\{k_t, c_t\}_{t=0}^{\infty}$ converging toward $(\bar{k}, \bar{c})$ starting from $k_0 >0$.
			Also, the steady state is a balanced growth path.
		\end{note}

		\newpage

		\onecolumn


		\begin{figure}
			\center
			\includegraphics[width=\textwidth]{_images/phase}
			\caption{Ramsey Phase Diagram}\label{fig:phase}
		\end{figure}

			\begin{figure}
				\center
				\includegraphics[width=\textwidth]{_images/ex1}
				\caption{Example 1}\label{fig:ex1}
			\end{figure}

			\begin{figure}
				\center
				\includegraphics[width=\textwidth]{_images/ex2}
				\caption{Example 2}\label{fig:ex2}
			\end{figure}
		\twocolumn


\end{document}