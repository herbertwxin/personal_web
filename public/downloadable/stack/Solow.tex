\documentclass[twocolumn, fleqn]{article}
\usepackage[margin=1in]{geometry}
\usepackage{fancyhdr}
\usepackage{graphicx}
\usepackage{amsmath}
\usepackage{hyperref}
\usepackage{enumitem}
\usepackage{amssymb}

% Adjust column separation and add vertical rule
\setlength{\columnsep}{25pt} % Increase space between columns
\setlength{\columnseprule}{0.6pt}
\setlength{\jot}{6pt} % Set the space between equation lines to 6pt

% Header and Footer Settings
\pagestyle{fancy}
\fancyhf{}
\lhead{Core Macro I}
\chead{Solow Model}
\rhead{Page \thepage}
\renewcommand{\headrulewidth}{0.4pt}

\begin{document}

	\title{Solow Model}
	\author{Herbert W. Xin}
	\date{\today}
	\maketitle

	\tableofcontents
	\thispagestyle{fancy}

	\section{Kaldor's Facts}
		There are six stylized ``facts'' in Kaldor's statement, whereas the 7th one is often added.
		\begin{itemize}
			\item Output per worker ($Y/L$) -- or labor productivity -- grows at a sustained rate
			\item Capital per worker ($K/L$) grows at a sustained rate
			\item Capital-output rati ($K/Y$) stays roughly constant
			\item Shares of labor and capital in national income remain nearly constant
			\item Real interest rate is very stable
			\item Wide difference in the rate of growth between different countries
			\item Output per capita ($y/L$) varies greatly across countries
		\end{itemize}
		At the time, economists are trying to figure out a model to fit the Kaldor's facts.
		The conclusion we draw from Solow model fits very well indeed with Kaldor's facts.
		However, since the 21th century, many of the Kaldor's has fallen apart, especially the share of labor and
		capital in national income has not remain constant.
		Capital, in fact, now occupies a higher share of the total income.

	\section{Assumptions}
		There are some assumptions we impose on the model.

		\subsection{Basic assumptions}
			\begin{itemize}
				\item Single consumption good produced, can be converted to capital one-for-one.
				\item The good is produced using 2 inputs: capital ($K$) and labor ($L$)
				\item Capital ($K$) depreciates at rate $0<\delta<1$
				\item Closed economy and no government -- all investment that happens is out of domestic saving
				\item Aggregate saving: $S=sY, 0<s<1$
				\item Markets are perfectly competitive -- this leads to indeterminate firm size
			\end{itemize}

		\subsection{Production function}
			The reason Solow model is classified as a Neoclassical model is because it uses a Neoclassical
			production function, which includes capital, labor and labor augmenting capital as input.
			Mathematically, this is
			\begin{equation}
				Y_t = F(K_t, A_t L_t)\label{eq:neoclassical_production}
			\end{equation}
			This means $F$ itself does not depend on time and time effect only comes through $K, A, L$.
			Also, as we stated earlier, $A$ is labor augmenting.

			\subsubsection{Assumptions about $F$}
				We assume positive and diminishing marginal product
				\begin{align*}
					\frac{\partial F}{\partial K} &\equiv F_k >0, \frac{\partial^2 F}{\partial K^2} <0\\[4pt]
					\frac{\partial F}{\partial L} &\equiv F_k >0, \frac{\partial^2 F}{\partial L^2} <0
				\end{align*}
				Also, $F_{K,L}>0$ ensures profit maximization leads to an optimal solution.

				On the other hand, $F$ satisfies constant return to scale, i.e.
				\begin{equation*}
					F(\lambda K, \lambda AL) = \lambda F(K, AL), \forall \lambda >0, \text{ i.e. } HD1
				\end{equation*}

				Suppose we set $\lambda = \frac{1}{AL}$, an effective unit of labor, gives
				\begin{align*}
					F(\underbrace{\frac{1}{AL}K}_{\text{capital per effective labor}}, \frac{1}{AL}AL) = F(k,1) \\
					\equiv f(k) = \frac{Y}{AL} = y
				\end{align*}

				The production function expressed in capital per effective is also called intensive form production as it is expressed of labor/factor intensity.

			\subsubsection{Inada conditions}
				The production function needs a strictly positive amount of both inputs produce any output, i.e
				\begin{equation*}
					F(0,AL) = 0 = F(K,0)
				\end{equation*}
				We also have $f(0)=0, f' >0, f''<0$

				Inada conditions also limit the asymptotic behavior
				\begin{align*}
					\lim_{K \rightarrow 0} F_K &= \infty, \lim_{K \rightarrow \infty} F_K= 0\\
					\lim_{L \rightarrow 0} F_L &= \infty, \lim_{L \rightarrow \infty} F_L = 0\\
					\lim_{k \rightarrow 0} f &= \infty, \lim_{k \rightarrow \infty} f= 0
				\end{align*}

			\subsubsection{Cobb-Douglas production function}
				For the majority of time, we use Cobb-Douglas production function, which satisfies Inada condition
				and constant return to scale.
				\begin{align*}
					Y &= B K^{\alpha}(AL)^{1-\alpha}, 0<\alpha<1
				\end{align*}

	\section{Solving the model}
		\subsection{Income shares}
			Let the rent rate of capital be $Q$, wage per worker be $W$, then capital's share of income is $\frac{QK}{Y}$, labor's share of income is $\frac{WL}{Y}$

			Now we try to solve for firm's profit maximization condition
			\begin{equation*}
				\max_{K_t,L_t} \Pi_t = P_t Y_t - W_t L_t - Q_t K_t
			\end{equation*}
			We normalize the price $P_t = 1, \forall t$, this allows us to express $W_t$ and $Q_t$ in the price
			of final goods, which simplifies the calculation and interpretation.

			First order condition gives
			\begin{align*}
				\frac{\partial \Pi_t}{\partial L_t} &= 0\\
				\implies \frac{\partial Y_t}{\partial L_t}-W_t&=0\\
				\implies \frac{\partial F(K_t,A_t L_t)}{\partial L_t}-W_t&=0\\
				\implies F_L(K_t,A_t L_t) &=W_t
			\end{align*}

			Similarily, \[Q_t = F_W(K_t,A_t L_t) \implies f'(k_t)=Q_t\]

			Now, if we express the wage function in terms of intensive labor:
			\begin{align*}
				W &= \frac{\partial A_t L_t f(k_t)}{\partial L_t}\\
				W &= A_t f(k_t) + A_t L_t \frac{\partial f(k_t)}{\partial k_t}\frac{\partial k_t}{L_t}\\
				&= A_t f(k_t) + f'(k_t) \frac{K_t}{L_t}\\
				&= A_t f(k_t) - A_t k_t f'(k_t)\\[4pt]
				&= A_t [f(k_t)-k_t f'(k_t)]
			\end{align*}

			Now suppose we also have a TFP term ($B$) in the production function, i.e.
			\[Y_t = B K_t^\alpha (A_t L_t)^{1-\alpha}\]
			Then the derived return to capital and wage would be
			\begin{align*}
				Q_t &= f'(k_t) = \alpha B_t ^{\alpha-1}\\[4pt]
				W_t &= A_t [f(k_t)-k_t f'(k_t)]\\
				&= A_t [B k_t^\alpha - \alpha B k_t^\alpha]\\
				&= (1-\alpha)A_t B k_t^{\alpha}
			\end{align*}

			With this, we are able to calculate the capital and wage share of income
			\begin{align*}
				\frac{Q_t K_t}{Y_t} &= \frac{Q_t \frac{K_t}{A_t L_t}}{Y_t \frac{1}{A_t L_t}}\\
				\frac{Q_t k_t}{y_t} &= \frac{\alpha B k_t^\alpha}{B k_t^\alpha} =\alpha\\[4pt]
				\frac{W_t L_t}{Y_t} &= 1-\alpha\
			\end{align*}
			From above, this is exactly why we call $\alpha$ the capital share of income and this aligns with
			Kaldor's facts.

		\subsection{Evolution of inputs}
			In the Solow model, only $K$ is endogenous, $A_t, L_t$ are both exogenous variables.
			In fact, $L$ grows exponentially at the rate $n\geq 0$, i.e.
			\[L_{t+1} = (1+n)L_t \implies L_t (1+n)^t L_0\]
			If take natural log,
			\[\ln L_t = t \ln(1+n)+\ln L_0 \approxeq nt + \ln L_0\],
			or small value of $n$
			$A_t$ grows at rate $g\geq 0$.
			\[A_{t+1}= (1+g)A_t \implies A_t = (1+g)^t A_0\]

		\subsection{Capital accumulation \& dynamics}
			The so-called ``Solow Growth Equation'' describes the capital accumulation process:
			\[K_{t+1} = s Y_t + (1-\delta)K_t \equiv K_t + (I_t -\delta K_t)\]

			Also, in a closed economy: $I_t = S_t = s Y_t$, we also have the initial condition $K_0 >0$ is
			given, while $K_t$ is determined by past investment.\\

			\[
			\begin{cases}
				A_{t+1} = (1+g)A_t, & A_0 >0\\
				L_{t+1} = (1+n)L_t, & L_0 >1
			\end{cases}\]\\

			Now we write $Y_t$ as production function
			\[K_{t+1} = s F(K_t, A_t L_t)+(1-\delta)K_t\]
			and we divide both side by $A_t L_t$
			\begin{align*}
				\frac{K_{t+1}}{A_t L_t} &= s f(k_t) + (1-\delta)k_t\\[4pt]
				\frac{K_{t+1}}{A_{t+1} L_{t+1}} \frac{A_{t+1} L_{t+1}}{A_t K_t} &=  f(k_t) + (1-\delta)k_t\\[4pt]
				(1+g)(1+n)k_{t+1} &=  f(k_t) + (1-\delta)k_t\\[4pt]
				k_{t+1} &= \frac{ f(k_t) + (1-\delta)k_t}{(1+g)(1+n)}
			\end{align*}

			For small value of $g, n$, $ng$ is very small and we can reasonably ignore it and we can define
			$z = n+g$ to simplify.

			\begin{equation}
				k_t = \frac{s f(k_t)+(1-\delta)k_t}{1+z}\label{eq:solowss}
			\end{equation}

			This meant $K_t$ does not have a steady state but a balanced growth path, but $k_t$ has a steady state.

			Now we can also denote $k_{t+1}$ as function of $k_t$
			\[k_t = \frac{s f(k_t)+(1-\delta)k_t}{1+z} \equiv G(k_t)\]

			To find the local stability of $G(k_t)$ we can take the first order Taylor expansion around the
			steady state and it depend on $G'(k_t)$
			For our standard Solow model, we have two steady states $0, k^\ast$.

			Since $G(k_t) = \frac{s f(k_t)+(1-\delta)k_t}{1+z}$,
			This implies \[G'(k_t) = \frac{1}{1+z}[sf'(0)+(1-\delta)]\]

			We first test the stability of $0$,
			\[G'(0) = \frac{1}{1+z}[sf'(0)+(1-\delta)] = \infty\]
			As from Inada condition, $f'(0) = \infty$ $G'(0)=\infty$, which implies $0$ is not a stable steady
			state.

			Now we check the local stability for $k^\ast$
			\[G'(k^\ast) = \frac{1}{1+z}[s f'(k^\ast)+(1-\delta)]\]
			We know from our steady state condition
			\begin{align*}
			(1+z)k^{\ast} &= s f(k^\ast) + (1-\delta)k^{\ast}\\
			\implies (1-\delta) &= (1+z) - s \frac{f(k^\ast)}{k^\ast}\\
			\implies G'(k^\ast) &= \frac{1}{1+z}\left[s f'(k^\ast)+(1+z) - s \frac{f(k^\ast)}{k^\ast}\right]\\[4pt]
			G'(k^\ast) &= 1+\frac{1}{1+z}\underbrace{\left[s f'(k^\ast) - s \frac{f(k^\ast)}{k^\ast}\right]}_{<0}<1
			\end{align*}

	\section{Steady State}
		\subsection{Balanced growth path}
			The steady state of Neo-classical model, including Solow, often characterized by ``balanced growth
			path''.

			Balanced Growth Path of the Solow model is a path along which all quantities
				${Y_t, C_t, S_t, L_t, K_t}$ grow at constant exponential rate can be 0 Balanced Growth Path of Solow

			In fact, under steady state, all variables expressed in per capita term is constant
			\begin{align*}
				k_t &= k^{\ast}, \forall t\\
				y_t &= f(k_t) = f(k^\ast)=y^{\ast}, \forall t\\
				s_t &= sy_t = sy^{\ast}, \forall t\\
				c_t &= (1-s)y_t = (1-s)y^{\ast}, \forall t
			\end{align*}

			On the other hand, all the per labor term growths at $g$, as the only growing part is $A_t$
			\begin{align*}
				\frac{K_t}{L_t} &= A_t k_t = A_t k^{\ast}\\
				\frac{Y_t}{L_t} &= A_t y_t = A_t y^{\ast}\\
				\frac{C_t}{L_t} &= A_t (1-s)y_t = A_t (1-s)y^{\ast}\\
				\frac{S_t}{L_t} &= A_t sy_t = A_t s y^{\ast}
			\end{align*}

			Finally, most of the aggregate term grows at $z$
			\begin{align*}
				K_t &= A_t L_t k_t = \underbrace{A_t L_t}_{\text{grows at } z} k^{\ast}\\
				Y_t &= A_t L_t y_t = \underbrace{A_t L_t}_{\text{grows at } z} y^{\ast}\\
				C_t &= (1-s)Y_t\\
				S_t &= sY_t
			\end{align*}
			while $L_t$ grows at $n$

			Since both $K_t$ and $Y_t$ grow at $z$, ratio $\frac{K_t}{Y_t}$ will remain constant, which again,
			corresponds to the Kaldor's facts.

		\subsection{Factor prices in a steady state}
			\begin{align*}
				W_t & = A_t [f(k_t)-k_t f'(k_t)]\\
				&= A_t [f(k^\ast)-k^\ast f'(k^\ast)]\\
				Q_t &= f'(k_t) = f'(k^\ast) = Q^{\ast}
			\end{align*}
			Since the only growing part in $W_t$ is $A_t$ so naturally it will grow at $g$, while $Q$ is just a
			constant.

			Now, interest rate (return from capital net of depreciation) is also a constant as
			\[r_t = Q_t -\delta \]

			Wage per effective worker is also a constant
			\begin{align*}
				w_t &=\frac{W_t}{A_t}=f(k_t)-k_t f'(k_t)\\
				w^{\ast} &= f(k^\ast) - k^{\ast} f'(k^\ast)
			\end{align*}

			We can see from our previous discussion Solow model satisfies Kaldor's facts 1--5.
			Below are some interesting further readings.

			\textit{Uzawa}(1961) -- \textbf{The steady-state growth theorem}\\
				Any neo-classical model that exhibit steady state must have labor augmenting technology.\\

			3-types of technological progress
			\[
			\begin{cases}
				Y_t = F(K_t, A_tL_t) \leftarrow \text{labor augmenting}\\
				Y_t = F(A_tK_t, L_t) \leftarrow \text{capital augmenting}\\
				Y_t = A_tF(K_t, L_t) \leftarrow \text{factor neutral}
			\end{cases}\]\\

			In fact, Acemouglu has shown that any technology selected by people will eventually become labor augmenting in the long-run.

			However, an interesting fact is that Cobb-Douglas production function does not belong to the
			3-types, as any kind of technology progress can be expressed in labor augmenting technology in
			Cobb-Douglas.

			\begin{align*}
				Y_t &= B K_t^{\alpha}(A_t L_t )^{1-\alpha}\\
				Y_t &= B (A_t K_t)^{\alpha}(L_t )^{1-\alpha}\\
				&= B A_t^{\alpha} K_t^{\alpha}(L_t )^{1-\alpha}\\
				&= B K_t^{\alpha} (\underbrace{A_t^{\frac{\alpha}{1-\alpha}}}_{\Gamma_t}L_t)^{1-\alpha}
			\end{align*}
			If $A_t$ grows at $g$ then $\Gamma_t$ grows at $\frac{\alpha}{1-\alpha}g$

			Similarily
			\begin{align*}
				Y_t &= B A_t K_t^{\alpha}L_t^{1-\alpha}\\
				&= B K_t^{\alpha}(\underbrace{A_t^{\frac{1}{1-\alpha}}}_{\Phi_t}L_t)^{1-\alpha}
			\end{align*}
			$\Phi_t$ grows at $\frac{1}{1-\alpha}g$

	\section{Transitional Dynamics}
		We can find the growth rate of $k_t$ using our Solow growth equation
		\[(1+z)k_{t+1}=s f(k_t)+(1-\delta)k_t\]
		subtract $(1+z)k_t$ from both side
		\begin{align*}
		(1+z)(k_{t+1}-k_t) &= s f(k_t)+(1-\delta-1-z)k_t\\[4pt]
		\frac{k_{t+1}-k_t}{k_t}&=\frac{s f(k_t)/k_t - (z+\delta)}{1+z}
		\end{align*}

		\subsection{Comparative statics \& dynamics}
			Suppose $s$ increases, since we know from steady state $sf(k^\ast)=(z+\delta)k^\ast$:
			\[s = (z+\delta)\frac{k^\ast}{f(k^\ast)}\]
			So $s\uparrow \implies \frac{k^\ast}{f(k^\ast)} \implies k^\ast \uparrow \implies y^\ast, c^\ast, s^\ast \uparrow$.\\

			However, this will have no effect on the long-run growth.
			In fact, policy (permanent increases in saving or investment rate) has no effect on long-run growth,
			only a level of output per worker.
			Thus, we say policy only has a level effect, not a growth effect.

		\subsection{Golden Rule}
			A natural question to ask, however, is when this economy reaches efficiency.
			Here we borrow the definition of Pareto efficiency, i.e. the economy is efficient if we cannot change it
			a way without making everyone worse off.

			In particular, we want to find an economy where everyone consumes the maximum possible.
			\[\max_{s} (1-s)f(k^\ast) \implies f(k^\ast)-sf(k^\ast)\]
			Since we know $sf(k^\ast)=(z+g)k^\ast$,
			\[\implies \max_{s} f(k^\ast) - (z+g)k^\ast\]

			FOC gives
			\begin{align*}
				f'(k^\ast) \frac{\partial k^\ast}{\partial s}-(z+\delta) \frac{\partial k^\ast}{\partial s}&=0\\
				[f'(k^\ast)-(1+\delta)] \frac{\partial k^\ast}{\partial s}&=0\\
				\implies f'(k^\ast)-(1+\delta) &=0
			\end{align*}

			The condition above is the golden rule condition for capital stock.

			Thus, economists can use the equation below to measure an economy's efficiency.
			\[f'(k_{GR})-\delta = z \simeq n+g\]

			Back to our steady state condition
			\begin{align*}
				S_{GR} f(k_{GR}) &= (z+\delta)k_{GR}\\
				\implies S_{GR} &= \frac{(z +\delta)K_{GR}}{f(k_{GR})}
			\end{align*}
			Back to our steady state condition
			\begin{align*}
				S_{GR} f(k_{GR}) &= (z+\delta)k_{GR}\\
				\implies S_{GR} &= \frac{(z +\delta)K_{GR}}{f(k_{GR})}
			\end{align*}

\end{document}