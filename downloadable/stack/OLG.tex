% Preamble
\documentclass[twocolumn, fleqn]{article}
\usepackage[margin=1in]{geometry}
\usepackage{fancyhdr}
\usepackage{graphicx}
\usepackage{amsmath}
\usepackage{hyperref}
\usepackage{enumitem}
\usepackage{amssymb}
\usepackage{tcolorbox}
\usepackage{stmaryrd}



% Adjust column separation and add vertical rule
\setlength{\columnsep}{25pt} % Increase space between columns
\setlength{\columnseprule}{0.6pt}
\setlength{\jot}{6pt} % Set the space between equation lines to 6pt
\setlength{\parskip}{3pt}

% Header and Footer Settings
\pagestyle{fancy}
\fancyhf{}
\lhead{Core Macro I}
\chead{OLG Model}
\rhead{Page \thepage}
\renewcommand{\headrulewidth}{0.4pt}
\newtcolorbox{note}[1][]{
	enhanced,
	colback=white,            % Background color of the box content
	colframe=black,           % Border color
	coltitle=white,           % Title text color
	colbacktitle=black,       % Background color of the title area
	fonttitle=\bfseries,      % Bold font for the title
	boxrule=1pt,              % Border thickness
	title=SIDENOTE,           % Title of the box
	sharp corners,            % Square corners
	before skip=10pt,         % Space before the box
	after skip=10pt,          % Space after the box
	left=5pt,                 % Left padding
	right=5pt,                % Right padding
	top=5pt,                  % Top padding
	bottom=5pt,               % Bottom padding
	width=\columnwidth,       % Box width matches column width
	#1                        % Allows for optional parameters
}

\begin{document}

	\title{Overlapping Generation Model}
	\author{Herbert W. Xin}
	\date{\today}
	\maketitle

	\tableofcontents
	\thispagestyle{fancy}

	\vspace{0.5cm}
	\hline
	\vspace{0.2cm}
	Overlapping generation model divides the population into two segments—old and young generation.
	Agents only live for two periods, young and old.
	Young agents work and save, old agents can retire, but they do not save.
	Non-zero labor income in old age disturbs the dynamics.

	\section{Assumptions}\label{sec:assumptions}
		\begin{itemize}
			\item Number of young people at $t = N_t$
			\item Growth of population $N_{t+1} = (1+n)N_t$
			\item Work time endowment vector: $(1,e), e \leq 1$
			\footnote{The first element is the working time for the young (
				normalized to 1), the second is the working time for old}
		\end{itemize}

		In the baseline model, there is no cost of supplying labor, i.e. people do not value leisure, meaning they
		will simply supply $(1,e)$.

		The olds do not work ($e=0$), meaning they live off their savings from the young period.

		Finally, the olds do not exhibit altruism, meaning they do not care about their children, and no
		generational wealth transfer.
		If the olds care about children enough, they are essentially infinitely forward-looking.
		In a sense that they live forever, since they care about their children and their children care about their
		children.
		The preference passes back to the first generation and forms dynasty preference.

	\section{Preference}\label{sec:preference}
		The lifetime utility\footnote{This type of utility function is called additively seperatable, i.e. $u$ in stages do not depend on each other} can be expressed as
		\begin{equation}
			U_t = u(c_{1t})+\beta u(c_{2t+1}), \ 0<\beta<1\label{eq:preference}
		\end{equation}
		where $\beta = \frac{1}{1+\rho}$ is the subjective discount rate, i.e. how people discount future utility
		subjectively.

		In contract, the real interest rate $R_{t+1}$ is the objective discount rate.

		$u$ is a homogenous utility function and $U$ is homothetic function, i.e. $MRS(C_1, C_2)= MRS(\lambda C_1, \lambda C_2)$, which implies the indifference curve will be parallel and the income expansion path is a straight line.

		\subsection{Inada Condition}
			This prefernece also follows the Inada condition as in the Solow model.\\
			\(
			\begin{array}{llll}
				& w >0 & = & u''<0\\
				\lim_{c\rightarrow 0} & u'(c) & = & \infty\\
				\lim_{c\rightarrow 0} & u'(c) & = & 0
			\end{array}
			\)

	\section{Production Function}\label{sec:production-function}
		We use the same neoclassical production function, but now $t$ stands for generation time.
		\begin{equation}
			Y_t = F(K_t, A_t L_t), \forall t \label{eq:production}
		\end{equation}
		where technology grows at $g$, i.e.
		\[A_{t+1}=(1+g)A_{t}, g \geq 0\]

		Since the factor and output markets are perfectly competitive
		\begin{align*}
			W_t &= A_t [f(k_t)-k_t f''(k_t)]\\
			Q_t &= f'(k_t) \\
			r_t &= Q_t - \delta = f'(k_t)-\delta
		\end{align*}
		where the last equation is the return to capital, meaning the real interest is, in fact, marginal product of
		capital minus depreciation rate.

		\begin{note}
			Many assume \(\delta = 1\) since most capital stocks lost their value given a generation of time, but we
			do not consider this in our explaination.
		\end{note}

	\section{Solving the Model}\label{sec:solving-the-model}

		We start with the simple case where the olds fully retire.

		\subsection{Full retirement ($e=0$)}\label{subsec:full-retirement}
			Since the olds fully retire, part of the income goes to consumption, and the other part goes to saving.
			We write the young household's decision problem as follows:
			\begin{equation}
				\max_{c_{1t}, c_{2t+1},s_t} U_t=u(c_1t)+\beta u(c_{2t+1})\label{eq:UMP}
			\end{equation}
			subject to
			\(
			\begin{cases}
				c_{1t}+s_{t} = W_t\\
				c_{2t+1} = (1+r_{t+1})s_t \equiv R_{t+1}s_t
			\end{cases}\)

			Households take $W_t, r_{t+1}$ as given.

			Note that $r_{t+1}$ is not realized until $t+1$, we assume rational expectation so agents have perfect
			foresight.

			In this simple version, $s\geq 0$ as the olds cannot repay in the second period, and since households
			are identical, no one will lend.

			\begin{note}
				Time consistent decision arises from separable utility function and non-time-varying subjective
				discount factor.
			\end{note}

			\subsubsection{Lifetime budget constraint}
				If $s_t \neq 0$, then from the budget constraint:
				\[s_t = \frac{c_{2t+1}}{R_{t+1}}\]
				substitute this into the other equation
				\[c_{1t}+\frac{1}{R_{t+1}}c_{2t+1}=W_t\]
				where $\frac{1}{R_{t+1}}$ is the price of $c_{2t+1}$ in terms of $c_1$.
				This is sometimes called intertemporal price, as if the agent gives up 1 unit of $c_1$, he can save
				1 unit more and get $R_{t+1}>1$ unit of $c_{2t+1}$ in the next period.

			\subsubsection{Optimization}
				Now we express the UMP in terms of $s_t$:
				\begin{equation}
					\max_{s_t} U_t=u(W_t - s_t)+\beta u(R_{t+1}s_t) \label{eq:UMP_S}
				\end{equation}

				The first order condition gives
				\begin{align*}
					-u'(W_t - s_t)&+R_{t+1}\beta u'(R_{t+1}s_t)=0\\
					R_{t+1}\beta u'(R_{t+1}s_t)&= u'(W_t - s_t)\\
					u'(c_{1t})&= \beta R_{t+1}u'(c_{2t+1})\tag{Euler Equation}
				\end{align*}
				The last equation is known as the Euler equation of consumption, where it suggests the marginal loss
				of utility from less consumption due to saving should equal to marginal utility gain from extra
				wealth in the next period.

				We can also calculate the MRS from $u'(c_{1t})dc_{1t}+\beta u'(c_{2t+1})dc_{2t+1}=0$
				\begin{align*}
					\frac{dc_{2t+1}}{dc_{1t}} &= -\frac{u'(c_{1t})}{u'(c_{2t+1})}\\
					\frac{u'(c_{1t})}{u'(c_{2t+1})} &= R_{t+1}
				\end{align*}

	
				Now if we replace consumption with saving in the Euler equation, we get
				\begin{align*}
					u'(W_t - s_t) &= \beta R_{t+1}u'(R_{t+1}s_t)\\
					\implies s_t &= \Phi(W_t, R_t+1)
				\end{align*}
				In fact, we can express $s_t$ as a function of wage and real interest rate.

				\begin{note}
					We can divide the effect of real interest rate into income and substitution effect, when $R_{t+1}$ goes up, income effect suggests both consumptions should go up, as higher interest rate made agents wealthier.
					So saving goes down.

					Substitution effect suggests consumption in the second period should go up, as it becomes
					cheaper, which means given $W_t$, $s_t$ goes up.

					For log utility, income and substitution effects perfectly offset each other and $s$ is
					independent of $R$, as long as $e=0$.

					Observe the Euler equation with log utility:
					\begin{align*}
						\frac{1}{W_t - s_t} &= \frac{\beta R_{t+1}}{R_{t+1}s_t}\\
						\implies s_t &= \frac{\beta}{1-\beta}W_t
					\end{align*}
					Saving does not depend on $R_{t+1}$ at all
				\end{note}

			\subsubsection{Equilibrium prerequisite}

				Here are some known dynamics

				\begin{array}{lllll}
					L_t & = & N_t & = & (1+n)N_{t-1}\\[2pt]
					& & A_t & = & (1+g) A_{t-1}\\[2pt]
					& & s_t & = & \frac{\beta}{1+\beta}w_t A_t
				\end{array}\\

				Goods market clearing implies
				\begin{align*}
					w_t &= f(k_t) - k_t f'(k_t)\\
					R_{t+1} &= 1+r_{t+1} = 1+f'(k_t)-\delta
				\end{align*}

				Thus, in equilibrium, we should see
				\[s_t = \frac{\beta}{1+\beta}A_t[f(k_t) - k_t f'(k_t)]\]

				Asset marking clearing implies \[K_{t+1}=s_t N_t\]
				here $s_t$ is the saving level per young household,
				not the saving rate as in Solow model.
				Since the olds do not save, the aggregate saving level $S_t = s_t N_t$.
				Note this is no different of saving \[K_{t+1} = I_t + (1-\delta)K_t = S_t + (1-\delta)K_t\]
				because the current period capital is just the saving of the previous young (the current old), which will be consumed by the olds.

				From this we can find out the \textbf{law of motion for capital}.

				\begin{align*}
					K_{t+1} &= s_t L_t\\
					&= \frac{\beta}{1+\beta}A_t L_t[f(k_t) - k_t f'(k_t)]\\
					\frac{K_{t+1}}{A_t L_t} &= \frac{\beta}{1+\beta}[f(k_t) - k_t f'(k_t)]\\
					(1+z)k_{t+1} &= \frac{\beta}{1+\beta}[f(k_t) - k_t f'(k_t)] \tag{\text{Law of Motion of Capital}}
				\end{align*}
				given $k_0 = \frac{K_0}{A_0 L_0}>0$

			\subsubsection{Cobb-Douglas production}
				Suppose we have Cobb-Douglas production function.
				\begin{equation}
					f(k) = Bk^{\alpha}\label{eq:cobb-douglas}
				\end{equation}
				This implies \[f(k_t) - k_t f'(k_t) = (1-\alpha)Bk^\alpha\]

				Substitute back into the law of motion of capital allows us to find the steady state.
				\begin{align*}
					(1+z)k_{t+1} &= \frac{\beta}{1+\beta}(1-\alpha)Bk^\alpha\\
					k^{\alpha - 1} &= \frac{1+z}{1-\alpha} \frac{1+\beta}{\beta} \frac{1}{B}\\
					k &= \left[ \frac{1+z}{1-\alpha} \frac{1+\beta}{\beta} \frac{1}{B} \right]^{\frac{1}{\alpha -1}}
				\end{align*}
				Similar to the Solow model, 0 is also a steady state, albeit an unstable one.

		\subsection{Old age earning ($e\neq 0$ with log utility)}\label{subsec:old-age-earning}

			In this section, we assume log utility, which put the household UMP as
			\[\max_{c_{1t},c_{2t+1}}\ln(c_{1t})+\beta \ln (c_{2t+1})\]
			subject to
			\begin{cases}
				c_{1t}+s_{t} = W_t\\
				c_{2t+1} = (1+r_{t+1})s_t \equiv R_{t+1}s_t
			\end{cases}
			Again, rewrite this in terms of saving gives \[\max_{s_t} U_t=\ln(W_t - s_t) + \beta \ln(R_{t+1}s_t)\]

			First order condition then implies
			\begin{align*}
				\frac{1}{W_t s_t} &= \frac{R_{t+1}\beta}{R_{t+1}s_t + e W_{t+1}}\\
				(1+\beta)R_{t+1}s_t &= \beta R_{t+1}W_t - \beta R_{t+1}s_t\\
				s_t &= \frac{\beta}{1+\beta}W_t - \frac{1}{1+\beta}\frac{e W_{t+1}}{R_{t+1}}
			\end{align*}
			Now we can see the effect of real interest rate is contained in the second term, which drops out if $e=0$, more specifically,
			\[\frac{\partial s_t}{\partial R_{t+1}} = \frac{e W_{t+1}}{R_{t+1}^2}\frac{1}{1+\beta}>0\]

			\begin{note}
				The opportunity cost of leisure.
				Suppose agents care about leisure, so the UMP is \[\max u(c,l), \ s.t. p_c c = w(1-l) \]
				Rearrange gives \[c+wl = w\]
				Here, $wl$ is the opportunity cost of leisure, the term also contains wealth effect, the sum of
				wealth and income effect is called full income effct.
			\end{note}

			Put this into lifetime budget constraint gives
			\[c_{1t}+\frac{c_{2t+1}}{R_{t+1}}=W_t + \frac{e W_{t+1}}{R_{t+1}}\]

			Now, when $R_{t+1}$ goes up, we see three effects
			\begin{enumerate}
				\item Income effect -- total income goes up, so consume more -- $s_t \downarrow$
				\item Substitution effect -- consumption tomorrow becomes cheaper -- $s_t \uparrow$
				\item Wealth effect -- PV of lifetime income is lower, save more against that to balance out utility
				-- $s_t \uparrow$
			\end{enumerate}

			With log utility and $e=0$, 1 and 2 cancel out and 3 does not exist, so the effect of real interest
			rate on saving is 0.

			With log utility and $e\neq 0$, however, 1 and 2 cancel out but 3 is present, so
			$\frac{\partial s_t}{\partial R_{t+1}}>0$

		\subsection{CES utility with $e=0$}\label{subsec:ces-utility}
			For CES utility function, we have
			\begin{equation}
				u(c) =
				\begin{cases}
					\frac{c^{1-\sigma}-1}{1-\sigma}, & \sigma >0\\
					\ln(c), & \sigma =1
				\end{cases}
				\label{eq:CES}
			\end{equation}
			For now, we use a simplified version of CES: \[u(c) = \frac{c^{1-\sigma}}{1-\sigma}\]

			\subsubsection{Optimal saving}
				We use the Euler equation again
				\begin{align*}
					u'(c_{1t}) &= \beta R_{t+1}u'(c{2t+1})\\
					c_{1t}^{-\sigma} &= \beta R_{t+1} c_{2t+1}^{-\sigma}\\
					\left( \frac{c_{2t+1}}{c_{1t}} \right)^{\sigma} &= \beta R_{t+1}\\
					1 &= \frac{1}{\beta R_{t+1}}\left( \frac{c_{2t+1}}{c_{1t}} \right)^{\sigma}
				\end{align*}

				Now we use the period 2 budget constraint
				\begin{align*}
					c_{2t+1} &= R_{t+1}s_t + eW_{t+1}\\
					s_t &= \frac{c_{2t+1}}{R_{t+1}} - \frac{eW_{t+1}}{R_{t+1}}\\
					s_t &= \beta^{1/\sigma} R_{t+1}^{1/\sigma -1}c_{1t}- \frac{eW_{t+1}}{R_{t+1}}\\
					s_t &= \beta^{1/\sigma} R_{t+1}^{1/\sigma -1}(W_t - s_t)- \frac{eW_{t+1}}{R_{t+1}}
				\end{align*}

				Since $e=0$
				\[s_t = \frac{\beta^{1/\sigma}R_{t+1}^{1/\sigma-1}W_t}{1+\beta^{1/\sigma}R_{t+1}^{1/\sigma-1}}W_t\]

				If $\sigma =1$
				\[s_t = \frac{\beta}{1+\beta}W_t \implies \frac{\partial s_t}{\partial R_{t+1}=0}\]

				If $\sigma <1$
				\[\frac{\partial s_t}{\partial R_{t+1}}>0, \text{ as } \frac{1}{\sigma}>1\]

				If $\sigma >1$
				\[\frac{\partial s_t}{\partial R_{t+1}}<0, \text{ as } \frac{1}{\sigma}<1\]

				The best part about CES utility function is that
				\begin{cases}
					\sigma \rightarrow 0  \implies & \text{ Leontieff}\\
					\sigma \rightarrow \infty  \implies & \text{ Linear}
				\end{cases}

				Higher value of $\sigma$ corresponds to less welling to substituting between $c_1 \& \ c_2$, i.e.
				lower intertemporal elasticity of substitution (IES $= \frac{1}{\sigma}$), while lower $\sigma
				$ implies higher intertemporal elasticity of substitution.

				We can write CES as \[u(c) = \frac{c^{1/\sigma-1}}{1/\sigma-1}\]
				and $\frac{1}{\sigma}$ is the IES.

	\section{General Equilibrium}\label{sec:general-equilibrium}

		Equilibrium is called equilibrium as they follow a certain law of motion from the equilibrium condition.

		A competitive equilirbium of an economy is a sequence of aggregate capital stocks, consumptions, and factor
		prices $\{K_{t+1}, c_{1t}, c_{2t+1}, R_t, W_t\}_{t=0}^{\infty}$ such that factor prices are given by
		$W_t = A_t[f(k_t)-f'(k_t)k_t]$ and $r_t = f'(k_t)-\delta$.
		Individual consumption decisions are governed by $u'(c_{1t})=\beta R_{t+1}u'(c_{2t+1})$, and the aggregate
		captal stock evolves according to $K_{t+1}=s_t N_t$.

		The aggregate law of motion for capital is
		\begin{equation*}
			K_{t+1} = Y_t - C_t + (1-\delta)K_t
		\end{equation*}
		This could be rewritten as \[K_{t+1}=(s_t^y+s_t^0)+(1-\delta)K_t\]
		where $s_t^y+s_t^0$ is the young plus old saving.

		Now this is just \[s_t N_t - (1-\delta)K_t + (1-\delta)K_t\]
		where the first term in saving of current young, the second term is the dissaving of current old, which
		cancels out with the depreciated capital stock from the last period.

		If we assume $e=0$ from now on, then we can express capital as 
		\[K_{t+1}=s_t N_t = s_t L_t = s(w_t A_t, R_{t+1})L_t\]
	
		In general equilibrium
		\begin{align*}
			w_t &= w(k_t) \\
			k_t &= 1 + \vartheta_{t+1} = R(k_{t+1})\\
			\implies K_{t+1} &= s(w(k_t)A_t, R(k_{t+1}))L_t\\
			\frac{K_{t+1}}{A_t L_t} &= \frac{s(w(k_t)A_t, R(k_{t+1}))}{A_t}\\
			(1+z)k_{t+1} &= \Psi(w(k_t),R(k_{t+1}))
		\end{align*}
		where the last equation strictly holds if we assume homothetic preference.
		Also, note this is just a difference equation in $k_t$, given $k_0 =\frac{K_0}{A_0L_0}$.

		Suppose we also have CES utility function, we have
		\[\Psi(w(k_t),R(k_{t+1}))=\frac{\beta^{1/\sigma}R_{t+1}^{1/\sigma-1}}{1+\beta^{1/\sigma}R_{t+1}^{1/\sigma-1}}w(k_t)\]
		also, we know $R(k) = 1+r(k)= 1+f'(k)-\delta, R'(k)<0$ for diminishing $MPK$.

		If $1/\sigma >1, \sigma <1$, then when $k_t$ goes up, $k_{t+1}$ does not go up by a lot, higher $k_{t+1} \implies R(k_{t+1})\downarrow
		$, which decreases saving propensity, so the effect on $k_{t+1}$ is dampened.

		If $1/\sigma >1, \sigma <1$, $k_t \uparrow \implies k_{t+1} \uparrow \implies k_{t+1} \uparrow \uparrow$.

		We will rule out multiple positive steady state by assume $\sigma \leq 1 (IES \geq 1)$.

		\subsection{The importance of IES and $e$}\label{subsec:importance-of-ies}

			A crucial note to take in mind is that $IES = \frac{1}{\sigma}$ affects the equilibrium dynamics
			dramatically.

			If $IES>1$, substitution effect outweighs income effect, when $IES<1$, then income effect outweighs
			substitution effect.

			Again, higher $IES$ means agents are more willing to trade between consumption to maximize utility.
			Thus, when $IES$ is high, the substitution effect outweighs as agents try to maximize utility with intertemporal
			substitution.

			When $IES$ is low, the incentive for intertemporal substitution is low, meaning substitution effect is low
			and the income effect dominates.

		\subsection{Equilibrium dynamics}\label{subsec:equilibrium-dynamics}

			Recall with Cobb-Douglas production function and $e = 0$, the steady state capital stock is
			\[k^\ast = \left[ \frac{1+z}{1-\alpha} \frac{1+\beta}{\beta} \frac{1}{B} \right]^{\frac{1}{\alpha -1}}\]

			With some calculus, it is easy to show
			\begin{align*}
				\frac{\partial k^\ast}{\partial z} &< 0\\
				\frac{\partial k^\ast}{\partial \beta} &> 0\\
				\frac{\partial k^\ast}{\partial B} &> 0
			\end{align*}

		\subsection{Balanced growth path}\label{subsec:balanced-growth-path}
			The steady state is only one type of equilibrium, not the only one.
			In fact, the steady state of the OLG model follows a similar balanced growth path as the Solow model.
			Recall in the assumption section we said $A_t$ grows at the rate of $g$ and $N_t=L_t$ growth at the rate
			of $n$.

			Thus, the aggregate level of output, consumption, and capital all grow at the level $z = n+g+ng$.
			Wage per worker grows at the rate of technology growth $g$.
			Whereas the output per effective work $y_t = f(k^\ast)$ remains constant.
			The same holds true for wage per effective worker ($w_t = w^\ast = w(k^\ast)$) and real interest rate
			$r_t = Q_t -\delta=Q(k^\ast)-r=r^\ast$.



	\section{Golden Rule}\label{sec:golden-rule}
		The economy resource constraint gives \[Y_t = C_t + I_t\], this can be rewrite as
		\[C_t + K_{t+1} - (1-\delta)K_t = F(K_t, A_t L_t)\]
		divide both side by $A_t L_t$ gives
		\[c_t + (1+z)k_{t+1}-(1-\delta)k_t = f(k_t)\]
		where $c_t$ is consumption per effective worker

		In steady state this becomes:
		\begin{align*}
			c &= f(k) -(1+z)k+(1-\delta)k\\
			c &= f(k) - (z+\delta)k
		\end{align*}

		Golden rule capital stock per effective worker maximizes $c$, so we take FOC w.r.t $k$
		\begin{align*}
			\frac{\partial c}{\partial k} &= 0\\
			f'(k_{GR}) &= z+\delta\\
			f'(k_{GR})-\delta &= z\\
			\vartheta_{GR} &= z
		\end{align*}

		For Cobb-Douglas production function
		\begin{align*}
			f(k) &= Bk^{\ast}\\
			f'(k_{GR}) &= \alpha B(k_{GR})^{\alpha-1}=z+\delta\\
			k_{GR} &= \left( \frac{\alpha B}{z+\delta} \right)^{\frac{1}{1-\alpha}}
		\end{align*}

		In contrast, \[k^\ast = \left[ \frac{1+z}{1-\alpha} \frac{1+\beta}{\beta} \frac{1}{B} \right]^{\frac{1}{\alpha -1}}\]
		In general, there is no reason for $k^\ast =k_{GR}$

		Most importantly, $k^\ast > k_{GR}$ is dynamically inefficient.
		Now, if we compare the two capital stock, for the economy to be dynamically inefficient, we need
		\begin{align*}
			\frac{1}{1+z} \frac{\beta}{1+\beta}(1-\alpha)B &> \frac{\alpha B}{z+\delta}\\
			\implies \frac{z+\delta}{1+z} \frac{\beta}{1+\beta} > \frac{\alpha}{1-\alpha}
		\end{align*}
		where $\frac{z+\delta}{1+z}$ increases in $z$, $\frac{\beta}{1+\beta}$ increases in $\beta$ and
		$\frac{\alpha}{1-\alpha}$ increases in $\alpha$.
		This condition is more likely to hold for higher value of $z$, $\beta$ and lower value of
		$\alpha$.

		Higher $\beta$ suggests higher patient and thus more saving.
		Lower value of $\alpha$ means higher proportion in $W_t$, as
		$W_t = (1-\alpha)A+Bk^\alpha$, combining the two leads to a higher propensity to save too
		much, i.e. more capital accumulation and higher $k$.

	\section{Dynamic Inefficiency}\label{sec:dynamic-inefficiency}
		The technology and population growth play an integral part of efficiency in OLG model.
		To understand the role of $z$, take a simple example of $g=0$, i.e. $z=n$

		In the steady state of competitive equilibrium, one unit of saving leads to $\vartheta$
		\footnote{Recall $\vartheta$ is the marginal product of capital minus depreciation} unit of future consumption.

		Now, if we switch to a social planner problem, where each young gives up one unit of consumption to the
		planner, which the planner redistributes to the current olds.
		This means current old gets $1+n$ units of consumption, and every young gets $1+n$ units of saving when they
		are old.

		The social planner scheme is better if $1+n > 1+r^\ast$, i.e. the wealth transfer provides more old age
		consumption than saving by self.
		Mathematically, this is \[r^\ast < n =z\]
		If this condition holds, then the economy is dynamically inefficient as the social planner to provide higher
		consumption for everyone.
		This relates back to the previous discussion in section \ref{sec:golden-rule} where high $n$ leads to dynamic
		inefficiency.

		The reason behind this has to do with the infinite number of generations.
		To see this, consider the economy's aggregate resources constraint.
		\[Y_0 + \frac{Y_1}{R_1}+\frac{Y_2}{R_1 R_2}+\ldots+\frac{Y_n}{\prod_{i=1}^n R_i}\]

		In steady state, we have
		\begin{align*}
			Y_t &= A_t L_t f(k^\ast)\\
			\implies Y_{t+1} &= (1+z) Y_t
		\end{align*}

		So the resource constraint in steady state is
		\[Y_0 + \frac{1+z}{1+r^\ast} Y_0 + \left( \frac{1+z}{1+r^\ast} \right)^2 Y_0+\ldots\]
		collect the term gives
		\[Y_0\left[ \frac{1+z}{1+r^\ast} + \left( \frac{1+z}{1+r^\ast} \right)^2 +\ldots \right]\]

		In order for the resource constraint to be bounded, i.e.
		\[Y_0\left[ \frac{1+z}{1+r^\ast} + \left( \frac{1+z}{1+r^\ast} \right)^2 +\ldots \right]\leq \infty\]

		We need \[\frac{1+z}{1+r^\ast}<1 \implies r^\ast >z\]
		which intrinsically implies dynamical efficiency.
		This also implies the social planner cannot improve the economy by running a Ponzi scheme.

	\section{Social Security}\label{sec:social-security}
		There are two types of social security
		\begin{itemize}
			\item Pay as you go (PAYG) -- youngs pay money to the government, and the government redistributes the
			money to the olds.
			\item Fully funded (FF) -- force youngs to save, earns interest, and serve as the income source when old.
		\end{itemize}

		\subsection{Pay as you go (PAYG)}\label{subsec:pay-as-you-go-(payg)}
			Suppose social security tax is lump sum and $T_t=T, \forall t$.
			Then the young household's decision problem is \[\max \ln(c_t)+\beta \ln (c_{2t+1})\]
			subject to
			\begin{cases}
				c_{1t}+s_t =W_t - T, & T \leq W_t\\
				c_{2t+1} = R_{t+1}s_t + b
			\end{cases}

			where $b$ is the benefit paid from social security program.
			$T$ is set to be less or equal to $W_t$ to ensure $c_{1t}\geq 0$.

			Now, the government's budget constraint is \[TN_t = b N_{t-1}, \forall t\]
			i.e. the social security contribution needs to be equal to social security payment at all $t$.
			This implies
			\begin{align*}
				T(1+n)N_{t-1} &= b N_{t-1}\\
				\implies b &= (1+n)T
			\end{align*}

			We can substitute this back into the budget constraint and solve household's UMP w.r.t $s_t$, which gives
			\begin{align*}
				s_t &= \frac{\beta}{1+\beta}W_t - \frac{T}{1+\beta}(\beta + \frac{1+n}{R_{t+1}})\\
				&= \frac{\beta}{1+\beta}W_t - T \left( \frac{\frac{1+n}{1+r_{t+1}}+\beta}{1+\beta} \right)
			\end{align*}

			The effect of such social security system depends on the relationship between $n$ and $r_{t+1}$,

			\begin{cases}
				n = r_{t+1} & \frac{\partial s_t}{\partial T} = -1\\
				n > r_{t+1} & \frac{\partial s_t}{\partial T} < -1\\
				n < r_{t+1} & \frac{\partial s_t}{\partial T} > -1
			\end{cases}

			In the first case, saving fall one-for-one with extra $T$.
			Saving falls more than one-for-one for extra $T$ in the second case and less than one-for-one in the
			third case.

			In equilibrium,
			\begin{align*}
				K_{t+1}&= s_t N_t \\
				(1+n)k_{t+1} &= s_t\\
				(1+n)k_{t+1} &= \frac{\beta}{1+\beta}W_t - \frac{T}{1+\beta}(\beta + \frac{1+n}{R_{t+1}})
			\end{align*}

			If we assume technology growth is 0, i.e. $A_t = A_0 = 1, \forall t$, we can rewrite the equation above as
			\[(1+n)k_{t+1} = \frac{\beta}{1+\beta}w(k_t) - \frac{T}{1+\beta}
			\left(\beta + \frac{1+n}{1+Q(k_{t+1})-\delta}\right)\]

			This implies $\frac{\partial k_t^\ast}{\partial T}<0$, which means the steady state saving rate
			decreases with the introduction of PAYG social security.

			It is natural for us to ask if this type of social security makes dynamically inefficient more efficient.
			The answer is \textbf{yes}, it does improve welfare.
			To see this, observe the lifetime budget constraint
			\[c_{1t}+\frac{c_{2t+1}}{R_{t+1}=W_t - T\left[ \frac{r_{t+1}-n}{1+r_{t+1}} \right]}\]

			If the economy was initially efficient, i.e. $k^\ast < k_{GR}$, obviously, this cannot be improve
			whatsoever.
			As increasing saving rate will always harm the current young.
			Same could be said for PAYG scheme, where the current old is better off while the the initial young are
			worse off.
			Mathematically, we have $k^\ast < k_{GR}$ and dynamic efficiency implies $r^\ast >n$, thus,
			\[\frac{r_{t+1}-n}{R_{t+1}}>0 \implies T \uparrow \implies c, s \downarrow \implies w \downarrow\]

			If the economy was initially inefficient, i.e. $k^\ast > k_{GR}, r^\ast <n$.
			Then, with the PAYG, both the olds and youngs are strictly better off.
			At the time of introduction, $k_t$ does not change.
			\[r^\ast <n, \frac{r_{t+1}-n}{R_{t+1}}<0 \implies T \uparrow \implies c \uparrow\]
			This means saving pays $r^\ast$ while social security pays $n$, subsequent generations are also strictly
			better off, although the gain will decreases and gradually oncverge to 0.

		\subsection{Fully Funded}\label{subsec:fully-funded}
			In fully funded regime, young household's decision problem is
			\[\max \ln(c_t)+\beta \ln (c_{2t+1})\]
			subject to
			\begin{cases}
				c_{1t}+s_t =W_t - T\\
				c_{2t+1} = R_{t+1}s_t + R_{t+1}T
			\end{cases}

			Solving for the UMP w.r.t $s_t$ yields \[s_t = \frac{\beta}{1+\beta}W_t -T, T\leq \frac{\beta}{1+\beta}W_t\]

			Now the law of motion of capital is
			\begin{align*}
				K_{t+1} &= s_t N_t + TN_t \\
				K_{t+1} &= (s_t + T)N_t \\
						&= \frac{\beta}{1+\beta}W_t N_t
			\end{align*}
			This means the fully funded scheme has no effect on capital accumulation, as it uses the tax collect to
			invest in the market, which means no inference on capital accumulation.
			The fully funded scheme will not be affected by demographic and time inconsistency either.

	\section{Intergenerational Altruism}
		There are two approaches to intergenerational altruism,
		\begin{enumerate}
			\item Dynastic preference: ``pure'' altruism
			\item ``Warm glow'' preference: impure altruism
		\end{enumerate}

		\subsection{Dynastic Preference}
			Take a two-period OLG model, the UMP of current households now becomes
			\[
				\max U_t = u(c_{1t}) +\beta u(c_{2t+1})+\gamma(1+n)U_{t+1}
			\]
			where
			\[
				U_{t+1} = u(c_{1t+1}) +\beta u(c_{2t+2})+\gamma(1+n)U_{t+2}
			\]
			subject to
			\begin{align*}
				c_{1t} &= W_t + b_t - s_t\\
				c_{2t+1} &= R_{t+1}s_t  +(1+n)b_{t+1}\\
				b_{t+1} &\geq 0, \text{ given } b_t \geq 0, (W_t, R_{t+1})
			\end{align*}

			We assume $\gamma(1+n)<1$ and $b_t$ is the inheritance received, $b_{t+1}$ is the bequest made.

			This effectively transforms households' utility function into an infinite horizon one.
			Using recursive substitution, we have
			\[
				\sum_{i=0}^{\infty} \{\gamma(1+n)\}^i \{u(c_{1t+i})+\beta u(c_{2t+i+1})\} +\lim_{s\rightarrow \infty}\{\gamma(1+n)\}^{s-t}U_s
			\]

			We need $\gamma(1+n)<1$ and sequence $\{U_s\}_{0}^{\infty}$ to be bounded above at $\bar{U}<\infty$,
			which gives
			\[
				\lim_{s\rightarrow \infty}\{\gamma(1+n)\}^{s-t}U_s =0
			\]

			Set up the two-period Lagrangian gives
			\begin{align*}
				\mathcal{L} &= u(c_{1t}) + \beta u(c_{2t+1})+ \gamma (1+n)
				[ u(c_{1t+1})\\
				&+ \beta u(c_{2t+2}) - \gamma (1+n) U_{t+2} ] + \lambda_t b_{t+1}
				\end{align*}
			where
			\begin{align*}
				c_{1t} &= w_t + b_t - s_t \\
				c_{2t+1} &= R_{t+1} s_t - (1+n) b_{t+1} \\
				c_{1t+1} &= w_{t+1} + b_{t+1} - s_{t+1}
			\end{align*}

			From the FOCs we have
			\begin{gather*}
				-u'(c_{1t}) + \beta R_{t+1} u'(c_{2t+1}) = 0 \tag{9.1}\\
				\gamma u'(c_{1t+1}) = \beta u'(c_{2t+1})- \frac{\lambda_t}{1+n} \tag{9.2}
			\end{gather*}

			And complementary slackness condition
			\[
				\lambda_t b_{t+1} = 0, \text{ with } \lambda_t =0 \text{ if } b_{t+1}>0
			\]

			Suppose we solve for an equilibrium where $b_t >0, \forall t$, the (9.2) becomes
			\[
				\gamma u'(c_{1t+1}) = \beta u'(c_{2t+1})
			\]

			If $\beta >\gamma$, then we have a selfish household
			\[
				 u'(c_{2t+1}) = \frac{\gamma}{\beta}u'(c_{1t+1}) < u'(c_{1t+1})
			\]

			Suppose we have CES utility function $u(c) = \frac{c_^{1-\sigma}}{1-\sigma}$, then we have
			\begin{align*}
				\frac{c_{2t+1}}{c_{1t}} &= (\beta R_{t+1})^{1/\sigma} \tag{9.3}\\
				\frac{c_{2t+1}}{c_{1t}} &= \left(\frac{\beta}{\gamma}\right)^{1/\sigma} \tag{9.4}
			\end{align*}

			Combining 9.3 and 9.4 gives
			\begin{align*}
				\frac{c_{1t+1}}{c_{1t}} &= \frac{c_{1t+1}}{c_{2t+1}} \cdot \frac{c_{2t+1}}{c_{1t}} = (\gamma R_{t+1})
				^{1/\sigma}\\
				\frac{c_{2t+1}}{c_{2t}} &=
				\frac{c_{2t+1}/(\beta/\gamma)^{1/\sigma}}{c_{2t/(\beta/\gamma)^{1/\sigma}}}=\frac{c_{1t+1}}{c_{1t}} = (\gamma R_{t+1})^{1/\sigma}
			\end{align*}

			Hence $\frac{c_{t+1}}{c_{t}}=(\gamma R_{t+1})^{1/\sigma}$, where $c_t = \frac{(1+n)c_{1t}+c_{2t}}{1+n}$
			is per capita consumption.

		\subsection{Warm-glow}
			Warm-glow, on the other hand, gives out a bequest in the last period, and households values the bequest
			in the last period.

			Now, this changes the UMP to:
			\[
				\max \ u(c_{1t}) + \beta u(c_{2t+1}) + \gamma (1+n) v(b_{t+1})
			\]

			subject to
			\begin{align*}
					c_{1t} &= w_t + b_t - s_t, \\
					c_{2t+1} &= R_{t+1}s_t - (1+n)b_{t+1}, \\
					b_{t+1} &\geq 0, \quad \text{given } b_t > 0
			\end{align*}

			Set up the Lagrangian gives:
			\[
				\mathcal{L} = u(c_{1t}) + \beta u(c_{2t+1}) + \lambda_t b_{t+1}
			\]

			First Order Conditions (FOC):
			\begin{gather*}
			    -u'(c_{1t}) + \beta R_{t+1} u'(c_{2t+1}) = 0\\
			    -(1+n)\beta u'(c_{2t+1}) + \gamma(1+n)v'(b_{t+1}) + \lambda_t = 0
			\end{gather*}
			
			Complementary Slackness:
			\[
				\lambda_t b_{t+1} = 0 \quad \text{with} \quad \lambda_t = 0 \ \text{if } b_{t+1} > 0
			\]

			If \( b_{t+1} = 0 \) and \( \lambda_t > 0 \), the second FOC implies:
			\[
				\beta u'(c_{2t+1}) > \gamma v'(b_{t+1})
			\]

			Similar to dynamic preferences.
\end{document}