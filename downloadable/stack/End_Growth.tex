%! Author = herbertxin
%! Date = 11/28/24

% Preamble
\documentclass[twocolumn, fleqn]{article}
\usepackage[margin=1in]{geometry}
\usepackage{fancyhdr}
\usepackage{graphicx}
\usepackage{amsmath}
\usepackage{hyperref}
\usepackage{enumitem}
\usepackage{amssymb}
\usepackage{tcolorbox}



% Adjust column separation and add vertical rule
\setlength{\columnsep}{25pt} % Increase space between columns
\setlength{\columnseprule}{0.6pt}
\setlength{\jot}{6pt} % Set the space between equation lines to 6pt
\setlength{\parskip}{3pt}

% Header and Footer Settings
\pagestyle{fancy}
\fancyhf{}
\lhead{Core Macro I}
\chead{Endogenous Growth}
\rhead{Page \thepage}
\renewcommand{\headrulewidth}{0.4pt}
\newtcolorbox{note}[1][]{
	enhanced,
	colback=white,            % Background color of the box content
	colframe=black,           % Border color
	coltitle=white,           % Title text color
	colbacktitle=black,       % Background color of the title area
	fonttitle=\bfseries,      % Bold font for the title
	boxrule=1pt,              % Border thickness
	title=SIDENOTE,           % Title of the box
	sharp corners,            % Square corners
	before skip=10pt,         % Space before the box
	after skip=10pt,          % Space after the box
	left=5pt,                 % Left padding
	right=5pt,                % Right padding
	top=5pt,                  % Top padding
	bottom=5pt,               % Bottom padding
	width=\columnwidth,       % Box width matches column width
	#1                        % Allows for optional parameters
}

\begin{document}

	\title{Endogenous Growth}
	\author{Herbert W. Xin}
	\date{\today}
	\maketitle

	\tableofcontents
	\thispagestyle{fancy}

	So far our discussion on output assumes technology is an exogenous process, i.e. $Y = \tilde{F}(A,K,L) = F(K,AL)$ where $F$ is CRS in $K$ and $L$.

	If $A$ increases endogenously, it is unrealistic to assume $F$ is CRS in $K,L$ and $A$.

	In perfectly competitive models, all factors $(A,K,L)$ paid their marginal products, which not enough income is
	generated to do that.
	To see this, consider the \textbf{Euler's Theorem}: If $F(A,K,L)$ is a homogeneous function of degree $\lambda>1$, then \[\tilde{F}_A A + \tilde{F}_K K + \tilde{F}_L L = \lambda \tilde{F} > \tilde{F} \]

	Four models and two solutions have been proposed to resolve this issue:
	\begin{enumerate}
		\item \textit{Romer (1986)} --- $A$ grows externally (not exogenously) to economic activity
		\item \textit{Romer (1990)} --- $A$ grows purposefully and markets are not competitive.
	\end{enumerate}

	\section{AK Model --- \textit{Romer (1986)}}\label{sec:ak-model}
		The \textit{Romer (1986)} model is so called ``AK Model'' because the aggregate technology is effectively
		linear in capital.
		This model introduces learning-by-doing (LBD) externality, which can be concluded in two points:
		\begin{itemize}
			\item production process generates knowledge externalities as a by-product
			\item since knowledge creation is accidental, no one needs to be compensated for it.
		\end{itemize}

		\subsection{Assumptions}\label{subsec:assumptions}
			There are two basic assumptions in this model:
			\begin{enumerate}
				\item \textbf{Learning-by-doing} works through each firm's investment. An increase in a firm's
				capital stock leads to an increase in its stock of knowledge $A_i$. The idea is supported more
				broadly by evidence that patents --- a proxy for learning --- closely follow investment in physical
				capital.
				\item \textbf{Knowledge is a public good} that any other firm can access at zero cost. This assumption
				implies that the change in each firm's technology $A_i$ corresponds to the economy's overall
				learning and is therefore proportional to the change in economy's average capital intensity.
			\end{enumerate}

		\subsection{Setup}\label{subsec:setup}
			The economy has a large number of small identical firms.

			$\bullet$ Firm $j$'s production function is CRS w.r.t its choice of capital and labor.
			\[Y_t^j = (K_t^{j})^\alpha (A_t L_t^j)^{1-\alpha} \implies y_t^j = A_t^{1-\alpha}(k_t^j)^\alpha\]
			where $y^j \equiv Y^j / L^j, k^j \equiv K^j / L^j$\\

			$\bullet$ At the aggregate level:
			\[A_t = \hat{A} k_t\]
			which embodies LBD externalities\\

			$\bullet$ Each $j$'s private marginal product:
			\begin{align*}
				\frac{\partial y_t^j}{\partial k_t^j}&=\alpha A_t^{1-\alpha} (k_t^j)^{\alpha-1}
				= \alpha (\hat{A} k_t)^{1-\alpha} (k_t^j)^{\alpha-1}\\
				&=\alpha A\left(\frac{k_t}{k_t^j}
				\right)^{1-\alpha}
			\end{align*}
			where $A \equiv \hat{A}^{1-\alpha}$\\

		\subsection{Equilibrium}\label{subsec:equilibrium}
			In a symmetric equilibrium $k_t^j = k_t \forall j$, then the private marginal product becomes
			\[\frac{\partial y_t^j}{\partial k_t^j}=\alpha A\]
			The symmetric equilibrium makes sense as all firms are small, identical firms, which should result in
			homogeneous solutions.
			If the solution is homogeneous, then individual solution is the same as the average.

			Social production function is
			\[y_t = (\hat{A}k_t)^{1-\alpha}k_t^\alpha=Ak_t\]

			Social marginal product of capital
			\[\frac{\partial y_t}{\partial k_t}= A\]

			The return to capital
			\[r_t = \frac{\partial y_t^j}{\partial k_t^j}\Big|_{k_t^j=k_t}-\delta=\alpha A -\delta\]

			Embed this production side in a Ramsey economy with CES utility function yields:
			\begin{align*}
				\left( \frac{C_{t+1}}{C_t} \right)^{\sigma} &= \beta (1+ r_{t+1})=\beta (1+\alpha A -\delta)\\
				\underbrace{\frac{c_{t+1}}{c_t}}_{1+g} &= \left[\beta (1+\alpha A -\delta)\right]^{1/\sigma}
			\end{align*}
			Now, as long as $\beta (1+\alpha A -\delta)>1 \implies g>0$, the economy grows forever.
			This requires $\beta$ and $A$ to be sufficiently large.
			Also, note that $C_{t+1}, C_t$ are both consumption per worker and $1+g$ is now endogenous.

			Contrast this to Neo-classical model, where
			\[1+g_t \equiv \frac{C_{t+1}}{C_t} =[\beta(1+f'(k_{t+1})-\delta)]^{1/\sigma} \]
			$g_t >0$ is not feasible since $\lim_{k\rightarrow \infty}f'(k)=0$ and $\beta(1-\delta)<1$.

			This means if we have a BGP with endogenous/unbounded growth, then $k_t$ needs to grow unboundedly, i.e. $\lim_{t \rightarrow \infty} k_t = \infty$, which contradicts with the initial assumption.


		\subsection{Takeaways}\label{subsec:takeaways}
			\begin{enumerate}
				\item The key to producing endogenous growth is the absence of diminishing return with respect to
				reproducible inputs, which is capital here (see next model for a more subtle take).
				\item The possibility of endogenous growth opens the door for policy to have a permanent effect on the
				growth rate
				\item Competitive equilibrium in this economy is not Pareto-optimal.
				\begin{itemize}
					\item Each $j$ ignores the effect of their choice of $k^j$ on $k$.
					\item Subsidy to increase private return on investment form $\alpha A$ to $A$.
				\end{itemize}
			\end{enumerate}

			To see the last point, suppose for every unit of $y^j$, the government provides a subsidy
				$(\frac{1-\alpha}{\alpha})$, which increases the net revenue to
				\[\left( 1+\frac{1-\alpha}{\alpha} \right)y^j = \frac{1}{\alpha}y^j\]

			These result $r = $ social return in competitive equilibrium.
			This subsidy can be funded through lump sum taxes on households, which meant Euler equation is unaffected
			but consumption growth increases to the optimal level.
			Thus, policy has a permanent effect on the growth rate. 
	
	\section{Jones-Mannelli Model --- \textit{Jones (1995)}}
		Jones model is also known as the semi-endogenous growth model.

		Suppose we break some of the Inada conditions in the production function:
		\[Y=F(K,L)=BK^\alpha L^{1-\alpha}+bK, \text{ where } b > \delta >0\]
		This implies
		\begin{align*}
			F(0,L) &=0, \text{ but } F(K,0)>0\\
			\lim_{K\rightarrow \infty}F_K &= \lim_{K\rightarrow \infty} (\alpha B K^{\alpha-1}L^{1-\alpha}+b)=b>0
		\end{align*}
		The underlining idea is that capital is vital for production and more productive than labor in a very
		specific way.

		Output per labor now becomes
		\[y \equiv \frac{Y}{L} = f(k) = Bk^\alpha + b K\]

		Embed this into the Ramsey economy with CES utility function yields
		\[r_t = f'(k_t)-\delta = \alpha B k_t^{\alpha-1}+b-\delta\]
		So the Euler equation becomes
		\[\frac{C_{t+1}}{C_{t}}= [\beta(1+\alpha B k_t^{\alpha-1}+b-\delta-\delta)]^{1/\sigma}\]

		If there is endogenous growth
		\[\frac{C_{t+1}}{C_{t}} \rightarrow [\beta(1+\alpha B +b-\delta-\delta)]^{1/\sigma}>1\]
		as long as $\beta (1+\beta -\delta)>1$, which happens with high $\beta$ and/or high $b$.

		\begin{note}
			Since endogenous growth is usually about the production side, the consumption/preference side is often
			not too relevant.
		\end{note}

	\section{Endogenous Growth with Human Capital --- \textit{Lucas (1988)}}
		The basic idea behind \textit{Lucas (1988)} model is that if the $Y$ is CRS in 2 reproducible inputs like
		physical and human capital, then the production function behaves like an AK technology.

		For example, if $h$ is humand capital per worker such that the production function becomes
		\[Y_t = BK_t^\alpha (h_t L_t)^{1-\alpha}\implies y_t \equiv \frac{Y_t}{L_t}=Bk_t^\alpha h_t^{1-\alpha} \]
		where $h$ grows via investment (net of depreciation)

		Embed this into a Solow economy
		\begin{align*}
			K_{t+1} &= s_K Y_t + (1-\delta_{k})K_t\\
			(1+n)k_{t+1} &= s_K y_t + (1-\delta_k)k_t
		\end{align*}

		Each worker augments their humand capital recording to
		\[h_{t+1} = s_H y_t + (1-\delta_H)h_t\]
		where the given initial conditions are $k_0>0, h_0>0$

		The two difference equations that determine the dynamic of the system are
		\begin{align*}
		(1+n)k_{t+1} &= s_K B k_t^{\alpha}h_t^{1-\alpha}+(1-\delta)k_t \tag{3.1}\\
			h_{t+1} &= s_H B k_t^{\alpha} h_t^{1-\alpha}+(1-\delta)h_t \tag{3.2}
		\end{align*}

		In the steady state (BGP) $k$ and $h$ are not constant, in fact, both grow at the same rate $g$, but $
		\frac{k}{h}$ is.

		Let $x_t \equiv \frac{k_t}{h_t}$, now if we divide (3.1) by $h_t$ gives
		\begin{align*}
		(1+n)\frac{k_{t+1}}{h_t} \frac{h_{t+1}}{h_{t+1}} &= s_K B \left(\frac{k_t}{h_t}\right)^{\alpha} + (1-\delta)\left( \frac{k_t}{h_t} \right)\\
		(1+n) x_{t+1} \frac{h_{t+1}}{h_t} &= s_K B x_{t}^{\alpha} + (1-\delta)x_t \tag{3.3}
		\end{align*}

		Repeat the same procedure for (3.2)
		\begin{align*}
			\frac{h_{t+1}}{h_t} = s_H Bx_t^{\alpha} +(1-\delta) \tag{3.4}
		\end{align*}

		Suppose $\frac{h_{t+1}}{h_t}=1+g = \frac{k_{t+1}}{k_t}, x_t = x \forall t$ in the BGP.
		Then from (3.3) we have
		\begin{align*}
		(1+n)(1+g)x &= s_K B x_^{\alpha} + (1-\delta)x\\
		(1+n)(1+g) &= s_K B x_^{\alpha-1} + (1-\delta) \tag{3.3a}
		\end{align*}

		From (3.4) we have
		\begin{align*}
		(1+g) &= s_K B x^{\alpha} + (1-\delta)x \tag{3.4a}
		\end{align*}

		Now, we have two equations and two unknowns in (3.3a) and (3.4a).
		Suppose we have $\delta=1$, then
		\begin{align*}
		(1+n)(1+g) &= s_K B x_^{\alpha-1}\\
		(1+g) &= s_H B x^{\alpha}
		\end{align*}

		Divide first by the second
		\begin{equation*}
			1+n = \frac{s_K}{s_H} \frac{1}{x} \implies x = \frac{s_K}{s_H} \frac{1}{1+n}
		\end{equation*}

		Substitute this into the second equation
		\begin{align*}
			1+g = \frac{B}{(1+n)^\alpha} s_K^{\alpha} s_H^{1-\alpha}>1
		\end{align*}
		Thus, this exhibits infinte growth under suitable restrictions, for example, large enough $B$ .

		Since in BGP, we have
		\[x = \frac{s_K}{s_H} \frac{1}{1+n} \implies h_t = \frac{s_H}{s_K}(1+n)k_t\]

		Substitute this into the production function
		\begin{align*}
			y_t &= B k_t^{\alpha}\left[ \frac{s_H}{s_K}(1+n)k_t \right]^{1-\alpha}\\
			 &= B \left( \frac{s_H}{s_K} \right)^{1-\alpha}(1+n)^{1-\alpha}k_t
		\end{align*}

	\section{Comparasion}
		To give a clear comparison between the first two models we discussed and the Ramsey, here we drop down the Euler
		equation for the three models.

		\begin{align*}
			&\text{Ramsey Model} (\sigma=1)\\
			&\frac{c_{t+1}}{c_t} = \beta (1+\alpha B k_{t+1}^{\alpha-1}-\delta)
			\tag{a}\\
			&\text{AK Model}\\
			&\frac{c_{t+1}}{c_t} = \beta (1+\alpha A -\delta) \tag{b}\\
			&\text{Jones-Mannelli Model}\\
			&\frac{c_{t+1}}{c_t} =
			\beta (1+\alpha B k_{t+1}^{\alpha-1}+b-\delta) \tag{c}
		\end{align*}

		\begin{figure}[htb]
			\center
			\includegraphics[width=0.5\textwidth]{_images/compare}
			\caption{Comparasion Chart}
		\end{figure}

	\section{Endogenous Growth through innovation}
		There are two approches to model technological progress:
		\begin{enumerate}
			\item Horizontal innovation --- \textit{Romer (1990)}: continuous expansion of input varieties used to
			manufacture the consumption good.
			\item Vertical innovation --- \textit{Aghion \& Howitt (1992)}: progressive imporvement in the quality of a
			limited number of intermediate goods.
		\end{enumerate}

		Here, we are going to focus on the \textit{Romer (1990)} model.

		\subsection{Setup}
			There are four sectors in \textit{Romer (1990)} model, but the household sector is trivial as it follows
			the setup of Ramsey, so the discussion below focuses on the rest three production sectors.

			\subsubsection{Final goods sector}
				The production function of the final good section is
				\begin{equation*}
				Y_t = L_{yt}^{1-\alpha}\left( \sum_{j=1}^{A_t} K_{jt}^\alpha  \right) \tag{5.1}
				\end{equation*}
				where $A_t$ is the number of intermediate input variesties $\{K_{jt}\}$ available at time $t$.
				$L_y$ is labor used in final goods production.
				The production function exhibits CRS and the final goods sector is perfectly competitive.

				Note $K_j$ are not perfect substitutes.
				Elasticity of substitution between any pair of $K_j$ is $\frac{1}{1-\alpha}$.
				$K_i$ dos not affect the marginal product of $K_j, j\neq i$, i.e.
				\[\frac{\partial Y_t}{\partial K_{jt}}= \alpha L_{yt}^{1-\alpha}K_{jt}^{\alpha-1}\]

				Also, there is diminishing marginal product for any $K_j$ that has already been invented.

			\subsubsection{R\&D Sector}
				The R\&D sector invents blueprints (``ideas'') for the production of new types of intermediate inputs.
				\begin{equation*}
					A_{t+1}-A_t = \theta A_t L_{At}, \theta>0, L_{At} = L_t - L_{yt} \tag{5.2}
				\end{equation*}
				$A_{t+1}-A_t$ is the flow of new blueprints, $A_t$ is the dynamc externality, i.e. accumulated
				knowledge makes it easier to invet new ideas.

				A new blueprint can be put into production with 1-period lag, i.e. ideas inveted in $t$ will lead to
				new $k$ varieties from $t+1$.

				R\&D firms sell blueprints to intermediate goods/input producers.

			\subsubsection{Intermediate Goods}
				A blueprint produced at $t$ leads to production of that $K_j$ from $t+1$ onwards.
				The sale of blueprints give the producer perpetual monopoly rights to produce that $K_{j}$.
				The market power comes from imperfect substitutability.

				Each $K_j$ produced 1 for 1 using the final and depreciates fully upon use (5.3)

		\subsection{Solving the Model}

		To solve the model, we assume Ramsey with CES utility, which gives the household side as
		\begin{equation*}
			\frac{c_{t+1}}{c_t} &= \left( \beta R_{t+1} \right)^{1/\sigma} \tag{5.4}
		\end{equation*}

		Now final goods producers produce $Y_t$ using $L_{yt}$ (hired at $W_t$) and $\{K_{jt}\}$ (produced at
			$\{P_{jt}\}$).
		Solving final goods producers profit maximization problem yields
		\begin{gather*}
			\max_{L_{yt}, \{K_{jt}\}} \Pi_t = L_{yt}^{1-\alpha}(\sum_{j=1}^{A_t}K_{jt}^\alpha )-W_t L_{yt}\\
			-\sum_{j=1}
			^{A_t}P_{jt}K_{jt} \\
			\implies (1-\alpha)L_{yt}^{-\alpha}(\sum_{j=1}^{A_t}K_{jt}^\alpha) = W_t \tag{5.5a}\\
			\implies \alpha L_{yt}^{1-\alpha}K_{jt}^{\alpha-1}=P_{jt}, \forall j \in \{1, \ldots, A_t\} \tag{5.5b}
		\end{gather*}

		The producer of $K_{jt}$ is a monopolist who understand the demand for their product is given by (5.5b).
		This gives the profit maximization problem below
		\begin{align*}
			\max_{K_{jt}} \pi_{jt} &= \underbrace{P(K_{jt})K_{jt}}_{(5.5b)}-\underbrace{K_{jt}}_{(5.3)}, \delta=1\\
			&= \alpha L_{yt}^{1-\alpha} K_{jt}^{\alpha}-K_{jt}
		\end{align*}

		FOC gives
		\[\alpha^2 L_{yt}^{1-\alpha}K_{jt}^{\alpha-1}= 1 \implies K_{jt}^\ast = \alpha^{\frac{2}{1-\alpha}}L_{yt}, \forall j \tag{5.6}\]

		Substitute this into (5.5b) gives
		\[P_{jt}^\ast = \frac{1}{\alpha} \tag{5.7}\]
		Since $\alpha<1$, this means $P_{jt}>1$, but the cost of producing $K_{tj}$ is 1, meaning there is a
			constant markup, so the maximized profit for intermediate firms are
		\begin{align*}
			\pi_{jt}^{\ast} &= P_{jt}^{\ast}K_{jt}^{\ast}-K_{jt}^{\ast}\\
			&= \frac{1-\alpha}{\alpha}\alpha^{2/(1-\alpha)} L_{yt} \tag{5.8}
		\end{align*}

		So there is a perpetual monopolist profit flow of $\{\pi_{jt}^\ast, \pi_{jt+1}^\ast, \pi_{jt+2}^\ast, \ldots\}$

		\begin{note}
			To simplify the assumption (and to get analytical solutions), suppose $L_t=L, \forall t$.
			We conjecture that in equilibrium
			\[L_{yt}=L_y, L_{At}=L_A = L-L_y, r_t = r, \forall t \]
		\end{note}

		Now the steam of profit becomes $\{\pi_{j}^\ast, \pi_{j}^\ast, \pi_{j}^\ast, \ldots\}$.

		However, recall the timing assumption, to earn this profit flow, i.e. monoply rents, the producer would have
			to pruchase the blueprint fro $K_j$ one peirod in advance.

		Let the price of a new blueprint at $t$ be $P_{At}$.
		Anyone can bid for such a blueprint.
		If there is free entry, then $P_{At}$ is the maximum bid someone makes.

		Note, the maximum bid equals to the PV of monopoly profits $t+1$ onwards, i.e.
		\begin{align*}
			P_{At}^j = P_A^j &= \frac{\pi_j^\ast}{1+r}\left[ 1+\frac{1}{1+r}+ \frac{1}{(1+r)^2} \right]\\
			&= \frac{\pi_j^\ast}{1+r} \frac{1}{1-\frac{1}{1+r}}= \frac{\pi_j^\ast}{r}
		\end{align*}

		From (5.8), we can see that we have a symmetric equilibrium, i.e. $\pi_j^\ast$ is the same $\forall j =\pi^\ast$.
		\begin{align*}
			\implies P_A^j = P_A &= \frac{\pi^\ast}{r} \forall j\\
			&= \frac{1}{r} \left( \frac{1-\alpha}{\alpha} \right) \alpha^{2/(1-\alpha)} L_y \tag{5.9}
		\end{align*}

		\begin{note}
			Even though $k$ producers earn monopoly profits, the PV of the net profit equals to 0 because of 9.
			This is because of the competitive aspect of the monopolistically competitive intermediate inputs
			sector, which also means zero equilibrium profits.
			Thus, the household budget constraint is unaffected by profits earned in this sector
		\end{note}

		We need to determine $L_y$ and $L_A$.
		Note first, labor is perfectly mobile across the two sectors, which implies equilibrium wages have to be the
			same.
		Both sectors hire in perfectly competitive labor markets
		\begin{gather*}
			VMP_L^Y = VMP_L^A\\
			\frac{\partial Y_t}{\partial L_t^Y} = P_{At} \frac{\partial (A_{t+1}-A_t)}{\partial L_t^A}\\
			(1-\alpha)L_{yt}^{-\alpha} \underbrace{(\sum_{j=1}^{A_t}K_{jt}^\alpha )}_{A_t \bar{K}_t^\alpha} = P_{At}\theta A_t\\
			(1-\alpha)L_{yt}^{-\alpha} \bar{K}_t^\alpha = \frac{1}{r}\left( \frac{1-\alpha}{\alpha} \right) \alpha^{
				2/(1-\alpha)}L_{yt}\theta\\
			L_y^{\alpha} \alpha^{\frac{2\alpha}{1-\alpha}}L_y^{\alpha} = \frac{\theta}{\alpha r}\alpha^{
				\frac{2}{1-\alpha}}L_y\\
			L_y = \frac{\alpha r}{\theta}\alpha^{-2}= \frac{r}{\alpha \theta} \tag{5.10}
		\end{gather*}
		Now if $r$ is constant, so is $L_y$ as conjectured.

		The last step is to determine $r$ and the economy's growth rate.
		From (5.4), we have
		\[1+g = \frac{c_{t+1}}{c_t}= (\beta R)^{1/\sigma}= [\beta(1+r)]^{1/\sigma} \tag{5.4'}\]
		From (5.2), we have
		\begin{align*}
			\frac{A_{t+1}-A_t}{A_t} &= \theta L_{At} = \theta(L-Ly)\\
			&= \theta(L -\frac{r}{\alpha \theta})\\
			\implies g &= \theta(L-\frac{r}{\alpha \theta}) \tag{5.2'}
		\end{align*}

		Solve (5.2') and (5.4') for $(r,g)$, neither equation depends on $t$, thus our conjecture of constant $(r,g)$ and $(L_y, L_A)$ verified.


		\subsection{Closed-Form Solution}
		Suppose $\sigma = 1$, i.e. log utility
		\begin{align*}
			r &= \frac{\alpha(1+\theta L -\beta)}{1+\alpha \beta}\\
			g &= \frac{\alpha \beta \theta L - (1-\beta)}{1+\alpha \beta}
		\end{align*}
		Note that $g>0$ only if $\theta L > \frac{1-\beta}{\alpha \beta}$, which can be interpreted as market size
			effect.
		Also, $\frac{\partial g}{\partial L}>0, \frac{\partial g}{\partial \beta}>0$.

		However, $g$ in efficiently low because of the two market failures
		\begin{enumerate}
			\item imperfect competition
			\item dynamic externality
		\end{enumerate}

		\subsection{Social Planner's Problem}
		To maximize the lifetime utility of the representative households efficiently, we try to solve the social
		planner's problem, subject to
		\begin{gather*}
			A_{t+1}- A_t = \theta A_t L_{At}\\
			Y_t = (L_t - L_{At})^{1-\alpha}(\sum_{j=1}^{A_t}K_{jt}^{\alpha})\\
			Y_t = C_t + A_{t+1} K_{t+1}
		\end{gather*}

		The result symmetric equilibrium
		\[g^P = \beta \theta L -(1-\beta) > \frac{\alpha \beta \theta L - (1-\beta)}{1+\alpha \beta}\]

		To counter the market failures,
		\begin{enumerate}
			\item Fixing the static inefficiency (imperfect competition)
			\begin{itemize}
				\item subsidize intermediate capital good production
				\item $K_j$ increases leads to $Y$ increase, but does not affect growth rate
			\end{itemize}
			\item Fixing the dynamic inefficiency (externality in knowledge production)
			\begin{itemize}
				\item tax labor in final goods production to allow more worker in R\&D sector
				\item resulting growth higher but less than $g^P$
			\end{itemize}
		\end{enumerate}
		We need the combination of both to achieve $g^P$


\end{document}